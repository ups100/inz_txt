\chapter{Architektura proponowanego systemu}
\label{chap:ArchCal}

\section[Podział na moduły][Podział na moduły]{Podział na moduły}

Monitorowanie klienta mobilnego jest zagadnieniem złożonym. Niestety
nie jest możliwe zaadoptowanie bez modyfikacji żadnego z~dostępnych na
rynku systemów. Wymagania przedstawione w~\ref{chap:Wymagania}
wymuszają budowę systemu monitorowania, który będzie umożliwiał
monitorowanie zarówno klienta mobilnego jak i~statycznego. Napisanie
całości takiego systemu jest zadaniem bardzo obszernym. Warto tutaj
nadmienić, że jądro monitorujące systemu Icinga zostało napisane
w~języku~C i~zajmuje 315~000 linii kodu. Na tej podstawie uznano za
niemożliwe napisanie systemu o~funkcjonalności szerszej od
wspominanego w~ramach pracy inżynierskiej.

Kolejnym argumentem przeciw tworzeniu takiego systemu od podstaw jest
konieczność przeprowadzenia obszernych testów. Większość systemów
dostępnych na rynku posiada rozbudowany system testów autoamtycznych,
a~także szeroką społeczność, co powoduje, że kod w nich zawarty jest
dobrze przetestowany, a~jego jakoś jest bardzo wysoka.

W obec powyższych, bezsprzecznie słusznych argumentów, w~ramach tej
pracy inżynierskiej zaproponowano system monitorujący oparty
o~rozwiązania dostępne na rynku, które zostały zmodyfikowane, aby
umożliwić monitoring klienta mobilnego. Proponowany system składa się
z~trzech modułów pełniących następujące funkcje:

\begin{itemize}
\item Moduł podstawowy, odpowiedzialny za bezpośredni monitoring
  klientów statycznych oraz analizę danych od klientów mobilnych.
\item Moduł odbioru danych, odpowiedzialny za przekazywanie wpisów
  dziennika od klientów mobilnych do modułu podstawowego.
\item Moduł mobilny, odpowiedzialny za monitorowanie klientów
  mobilnych i~przekazywanie danych do modułu odbioru danych.
\end{itemize}

%dodac rysunek z tymi komponentami
%dodac referencje do wiki icingi
Elementy modułu podstawowego rozmieszczone są na serwerach sieci
lokalnej, w~której znajdują się klienty statyczne. Moduł ten zapewnia
ich monitorowanie oraz stanowi interfejs dostępowy dla
administratora. W~przedstawionej kofiguracji istnieje tylko jedna
instancja jądra monitorującego, lecz możliwa jest również konfiguracją
zawierająca kilka rdzeni systemu monitorowania. Sposób wykonania
takiej kofiguracji został omowiony w~[wiki\_icingi]. Moduł odbioru
danych umieszczony jest na urządzeniu posiadającym dostęp do sieci,
w~której pracują klienty mobilne. W~omawianej kofiguracji przyjęto, iż
klienty mobilne oraz moduł odbioru danych mają dostęp do sieci
Internet. Ponadto w~celu uproszczenia omawianej kofiguracji
i~pominięcia dodatkowych elementów, które nie są przedmiotem tej
pracy, poczyniono założenie, iż moduł odbierający dane oraz rdzeń
monitorujący modułu podstawowego znajdują się na tym samym systemie
operacyjnym. Możliwa jest jednak konfiguracja, w~której wspomniane
elementy znajdują się na różnych urządzeniach. Moduł mobilny
instalowany jest na urządzeniu mobilnym, które jest monitorowane przez
system. Jest on bezpośrednio odpowiedzialny za wykonywanie
zaplanowanych odczytów oraz przekazanie ich wyników do modułu
odbierającego. Poszczególne elementy zostały omówione w kolejnych
podrozdziałach. Moduły omawianego systemu są od siebie niezależne
i~komunikują się poprzez dobrze zdefiniowane interfejsy
i~protokoły. Możliwa jest zatem zmiana przedstawionej konfiguracji,
w~taki sposób, aby spełniała ona dodatkowe oczekiwania odbiorcy.

\section[Moduł podstawowy][Moduł podstawowy]{Moduł podstawowy}

Moduł podstawowy stanowi rdzeń całego systemu monitorowania. Moduł ten
został zbudowany wykorzystując system monitorowania Icinga. System
monitorujący Icinga został szeroko opisany
w~\ref{subsec:Icinga}. Cieszy się on uznaniem środowiska
administratorów, a~jego możliwości konfiguracji umożliwiają budowę
rozległego systemu monitorowania rozproszonego zarówno dla bardzo
dużej sieci, jak i~dla dużej liczby klientów mobilnych.

Tutaj bedzie opis tego modulu

\section[Moduł odbioru danych][Moduł odbioru danych]{Moduł odbioru danych}
\label{sec:ModOdbDanych}

Moduł ten odpowiedzialny jest za odbieranie danych od klientów
mobilnych i~przekazywanie ich do modułu podstawowego. Żaden
z~dostępnych na rynku systemów monitorowania, ani dodatek do takiego
systemu, nie spełniał wymagań stawianych przed omawianym
systemem. W~związku z~powyższym moduł odbioru doanych został
zaprojektowany oraz zaimplementowany w~ramach niniejszej
pracy. Omawiany moduł jest niezależny od pozostałych. Możliwe jest
jego użycie zarówno z~innymi klientami mobilnymi, jaki z~innym
systemem monitorującym. Należy jedynie zapewnić implementacje
protokołu komunikacyjnego po stronie klienta, oraz dostarczyć metodę,
wprowadzania danych do modułu podstawowego. Omawiany moduł
przeznaczony jest dla systemów z~rodziny Linux. Składa się on
z~samodzielnego demona systemowego, którego szczegółowa architektura
została omówiona w~\ref{chap:ArchDaemona}.

%dodac odniesienia do literatury o AES i CBC
Moduł ten spełnia kilka bardzo ważnych funkcji. Podstawowym jego
zadaniem jest odbieranie danych od klienta. Zgodnie z~wymaganiami
określonymi w~\ref{chap:Wymagania} konieczne jest, aby system
zapewniał zarówno poufność jak i~integralność danych. Oba te wymagania
są spełnione przez ten system, poprzez użycie kryptografi oraz funkcji
skrótu, które pozwalają jednoznacznie zweryfikować integralność
wiadomości. System jest niezależny od zastosowanego algorytmu
szyfrowania danych. W omawianej konfiguracji, do transportu danych
został wykorzystany algorytm AES pracujący w trybie wiązania bloków
zaszyfrowanych (CBC). Jako długość klucza przyjęto 128 bitów, pomimo,
iż jest to najmniejsza z~dostępnych długosci klucza AES jest ona
wystarczająca do tego zastosowania.

Kolejnym zadaniem omawianego modułu jest uwierzytelnienie klienta. Dla
celów demonstracyjnych dostarczono dwa moduły
uwierzytelnienia. Pierwszy z~nich to uwierzytelnienie zawsze
pozytywne, które pozwala na dostęp każdemu klientowi. Drugi natomiast,
to proste uwierzytelnienie na podstawie nazwy użytkownika oraz hasła
przydzielonego przez administratora\footnote{Przedstawione moduły mają
  charakter czysto akdemicki i~służą jedynie zaprezentowaniu
  niezalezności modułu odbierania danych od wybranej metody
  uwierzytelnienia użytkownika. Wszystkie nazwy użytkowników oraz
  hasła, używane przez jedną z~metod przechowywane są jawnym
  tekstem. Przed użyciem systemu należy skonfigurować metodę
  uwierzytelnienia urzytkownika zgodnie z polityką stosowaną w danej
  sieci.}. Możliwe jest skonfigurowanie dowolnej motody
uwierzytelnienia klienta, zgodnie z polityką bezpieczeństwa stosowaną
w~danej sieci.

Odebrane od klienta dane są przechowywane przez moduł z~wykorzystaniem
pamięci trwałej, co umożliwa bardzo szybkie potwierdzenie odebrania
danych, jeszcze przed przekazaniem ich do miejsc
przeznaczenia. Omawiany moduł, na podstawie pliku konfiguracyjnego
dostarcza dane do wskazanych miejsc docelowych. Możliwe jest
definiowanie dowolnych miejsc przeznaczenia dla danych, co umożliwia
przekazywanie danych pochodzących od klienta do wielu systemów bez
konieczności ich retransmisji. Zastosowany plik pośredni gwarantuje,
iż czas zapisu danych w~miejsce docelowe, nie będzie wydłużał czasu
oczekiwania na potwierdzenie przyjęcia danych od klienta. Warto
zauważyć również, że dane odebrane przez ten moduł zostaną zawsze
dostarczone do ich miejsc docelowych, lub jawnie usuniętę przez
administratora. Oznacza to zatem, iż moduł gwarantuje dostarczenie
danych, natomiast wykrywanie duplikatów danych jest wykonywane
w~miejsca docelowych dla danych.

%Trzeba bedzie dopisac o walidacji danych jesli faktycznie bedzie tutaj robiona

\section[Moduł mobilny][Moduł mobilny]{Moduł mobilny}

Moduł ten jest odpowiedzialny za monitorowanie zadanych parametrów
urządzenia mobilnego. Każdy klient mobilny posiada swoją instancję
tego modułu, która jest odpowiedzialna za monitorowanie jego
urządzenia. Ten elemento systemu powinien posiadać budową
modularną. Najważniejsze z wymagań odnoszocych się do tego modułu
wymusza, aby możliwe było w jak najprostszy sposób dodawanie
możliwości sprawdzania nowych parametrów.  

Ponad to implementacja tego modułu musi brać pod uwagę architekturę
sprzętową na której pracuje. Urządzenia mobilne są zazwyczaj zasilane
z~własnych akumulatorów dlatego konieczne jest zastosowanie
mechanizmów, które pozwolą na zredukowanie zużycia energii związanego
z~systematycznym wykonywaniem sprawdzeń. Należy równiez wspomnieć, iż
moduł mobilny odpowiedzialny jest za nadawianie każdemu z~odczytów
stempla czasu uniwersalnego dokonywanego pomiaru. Na podstawie
dokonanej charakterystyki klienta mobilnego, w~niniejszej pracy
poczyniono założenie, iż klient posiada dostęp do punktów
synchronizacji czasu. Jest wiele dostępnych metod synchronizacji czasu
na urządzeniu mobilnym, midzy innymi pobranie czasu z~sieci GSM czy
też z~serwerów czasu światowego, przez co nie stanowi to dla klienta
mobilnego poważnego wymagania.

Klient mobilny po zebraniu porcji wpisów dziennika o~rozmiarze zgodnym
z~polityką administratora, lub po upływie określonego czasu powinien
przesłać posiadane wpisy dziennika do modułu odbiorczego, a~po
uzyskaniu potwierdzenia usunąć je z~urządzenia w~celu oszczędności
pamięci. Różnorodność platform dostępnych na rynku sprawia, iż nie
jest możliwe dostarczenie uniwersalnej implementacji protokołu
komunikacyjnego dla wszystkich platform mobilnych. Oznacza to, iż
przed klientem mobilnym stawia się wymóg zarówno implementacji
protokołu komunikacyjnego jak i~odpowiednich mechanizmów
uwierzytelnienia klienta. Protokół komunikacyjny został szczegółowo
opisany w~\ref{chap:ProtKom}.

%dodać referencje do pracy kubika
W ramach omawianego systemu wykorzystano wiele instancji modułu
mobilnego zaprojektowanego i~zaimplementowanego przez Pana Marcina
Kubika. Moduł ten został szeroko omówiony w [praca\_kubika]. Całość
systemu jest niezależna od platformy klienta mobilnego, zatem mozliwa
jest współpraca całości systemu z~klientami mobilnymi przeznaczonymi
dla innych platform, jednak muszą one implementować protokół
komunikacyjny wykorzystywany przez moduł pośredniczący.
