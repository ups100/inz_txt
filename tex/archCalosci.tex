\chapter{Architektura proponowanego systemu}
\label{chap:ArchCal}

\section[Podział na moduły][Podział na moduły]{Podział na moduły}

Klient mobilny, zdefiniowany w \ref{chap:Wymagania} jest urządzeniem,
co do którego, nie można zakładać, że powinno mieć nieprzerwany dostęp
do sieci internet. Ponadto nalezy zauważyć zmienność zarówno
geograficznego miejsca użytkowania jak i zmienność, wykorzystywanej
infrastruktóry sieciowej. Dodatkowo, należy odnieść się do wymagań, w
których zawarta jest konieczność minimalizowania zużycia energii przez
klietna mobilnego. Ciągłe utrzymywanie połączenia z serwere,
powodowałoby znaczne zużycie energii. Współpraca klienta mobilnego z
infrastrukturą publiczną nie pozwala również, na założenie, iż klient
mobilny posiada globalny adres IP\footnote{Globalny adres IP - adres
  protokołu internetowego działającego w warstwie sieciowej,
  pozwalający na unikalną identyfikację urządzenia w ramach całej
  sieci Internet.}. Powoduje to brak możliwości odpytywania klienta
mobilnego o jego stan.

Brak możliwości odpytywania klienta mobilnego o jego stan, wymusza
istnienie elementu systemu, znajdującego się, na urządzeniu
mobilnym. Element ten musi zatem minotorować urządzenie, na którym się
znajduje, a następnie, gdy pojawi się taka możliwość przekazywać dane
do podsystemu centralnego. Przekazanie danych powinno odbyć się w
sposób zapewniający poufność i integralność przesyłanych
danych. Konieczne jest również uniemożliwienie fałszowania danych
przez inne urządzenia oraz ich przechwycenenia. Tak postawione
wymagania w kwestii bezpieczeństwa powodują konieczność istnienia
elementu odpowiedzialnego za odebranie w sposób bezpieczny danych z
sieci zewnętrznej, a następnie przekazanie tych danych do podsystemu
odpowiedzialnego za ich właściwe przetwarzanie.

System powinien również umożliwiać monitorowanie infrastruktury
statycznej. Niezbędne, jest również udostępnienie danych bierzącyhc
administratorowi. Kolejnym z wymagań jest możliwość prezentacji i
analizy danych historycznych dotyczących wszystkich rodzajów
klientów. Konieczne jest zatem istnienie elementu systemu, który jest
odpowiedzialny, za monitorowanie klientów statycznych, a także
umożliwi przetwarzanie danych dostarczonych przez inne moduły od
klientów mobilnych. 

System, który spełni wymagania postawione w \ref{chap:Wymagania}
powinien składać się conajmniej z poniższych modułów:

\begin{itemize}
\item Moduł podstawowy, odpowiedzialny za bezpośredni monitoring
  klientów statycznych oraz analizę danych od klientów mobilnych.
\item Moduł odbioru danych, odpowiedzialny za przekazywanie wpisów
  dziennika od klientów mobilnych do modułu podstawowego.
\item Moduł mobilny, odpowiedzialny za monitorowanie klientów
  mobilnych i~przekazywanie danych do modułu odbioru danych.
\end{itemize}

%dodac rysunek z tymi komponentami

Dobre praktyki programistyczne, a także dbałość o możliwości rozwoju
systamu, nakazują umożliwienie komunikacji, pomiędzy modułami poprzez
dobrze zdefiniowane interfejsy. Zapewni to wymienność poszczególnych
modułów systemu i pozwoli na lepsze dostosowanie całego systemu do
oczekiwań konkretnego klienta. Należy zatem zwrócić szczególną uwagę,
na zapewnienie bezpiecznego i elastycznego protokołkołu komunikacji
pomiędzy klientem mobilnym, a modułem odbioru danych. Ważna jest
również komunikacja, pomiędzy modułem odbioru danych, a modułem
podstawowym. Należy ją zorganizować w sposób, który umożliwi
współpracę z różnymi, modułami podstawowymi.


\section[Moduł podstawowy][Moduł podstawowy]{Moduł podstawowy}

Moduł podstawowy stanowi rdzeń całego systemu monitorowania. Moduł ten
został zbudowany wykorzystując system monitorowania Icinga. System
monitorujący Icinga został szeroko opisany
w~\ref{subsec:Icinga}. Cieszy się on uznaniem środowiska
administratorów, a~jego możliwości konfiguracji umożliwiają budowę
rozległego systemu monitorowania rozproszonego zarówno dla bardzo
dużej sieci, jak i~dla dużej liczby klientów mobilnych.

Tutaj bedzie opis tego modulu

\section[Moduł odbioru danych][Moduł odbioru danych]{Moduł odbioru danych}
\label{sec:ModOdbDanych}

Moduł ten odpowiedzialny jest za odbieranie danych od klientów
mobilnych i~przekazywanie ich do modułu podstawowego. Stawiane
wymagania, nakładają na ten moduł możliwość obsługi wielu klientów,
używających wielu różnych platform. Ze względu na wymianę danych z
klientem mobilnym poprzez sieć Internet, moduł ten musi umożliwiać
bezpieczną wymianę danych. Szeroko rozumiany model bezpieczeństwa
realizowany przez ten moduł składa się z nastpujących elementów:

\begin{itemize}
\item poufność: przekazywane dane mogą zawierać tajemnice handlowe
  firmy oraz dane prywatne pracownika, dlatego ich transmisja powinna
  być szyfrowana. Wykorzystanie kryptografii asymetrycznej wiąże się z
  przechowywaniem dużej liczby kluczy klientów, a także wymaga od
  klienta mobilnego większej ilości obliczen. W związku z powyższym
  powinny zostać wykorzystane, algorytmy symetryczne o kluczach
  generowanych każdorazowo dla nawiązywanego połączenia, aby nie było
  konieczności przechowywania ich na urządzeniu mobilnym.

\item integralność: przekazywane dane powinny być przyjmowane tylko
  jeśli jest pewność, iż nie zostały one zmodyfikowane przez stronę
  trzecią. Aby to uzyskać, stosuje się funkcję skrótu, której wynik
  jest dołączany do wiadomości. Konieczne jest zatem, żeby moduł
  odbioru danych potrafił obliczać i weryfikować funkcje skrótu klasy
  co najmniej SHA-2.

\item uwierzytelnienie klienta: konieczność komunikacji poprzez sieć
  globalną, stwarza ryzyko odbioru danych od nieuprawionych
  urządzeń. Konieczne jest zatem, aby moduł ten umożliwiał
  przeprowadzenie weryfikacji tożsamości klienta. Zgodnie z
  wymaganiami, proces weryfikacji, powinien być niezależny od tego
  modułu i musi być możliwe definiowanie nowych, dowolnych metod
  uwierzytelnienia klienta.

\item uwierzytelnienie serwera: klient mobilny korzysta z różnych
  infrastruktur sieciowych. Część z nich może być narażona na ataki z
  zewnątrz. Konieczne jest zatem, aby moduł mobilny umożliwiał
  uwierzytelnienie się klientowi, czyli zapewnienia, że klient
  połączył się z autentycznym serwerem.

\item kontrola odbieranych danych: urządzenia mobilne, narażone są na
  nieautoryzowany dostęp. Konieczne jest, aby moduł odbiorczy
  kotrolował otrzymywane dane i przyjmował tylko te, które klient jest
  uprawiony przesyłać.

\end{itemize}


Zapewnienie bezpieczeństwa odbieranych nie jest jedynym zadaniem
modułu odbiorczego. Dane po odebraniu i odpowiedniej weryfikacji
powinny być przekazane do zdefiniowanych przez administratora
aplikacji. Konieczne jest, aby moduł odbiorczy umożliwiał przekazanie
danych do dowolnej aplikacji, w tym w szczególności, do wielu z nich
jednocześnie. Przekazywanie dancyh do wielu miejsc może być widziane
przez klienta jako znaczne opóźnienie w realizacji odbioru jego
danych. Należy zatem minimalizować czas pomiędzy odebraniem danych, a
potwierdzeniem ich przetworzenia. Gdy moduł odbiorczy dokona
potwierdzenia przetworzenia danych, jego obowiązkiem jest zapewnienie
ich fizicznego dostarczenia do zdefiniowancyh miejsc docelowych. Brak
możliwosci dostarczenia danych do miejsca docelowego powinien zostać
zaraportowany jako błąd, a dane przechowane na potrzeby późniejszej
synchronizacji.

Obecnie, na rynku dostępne są dodatki do systemów monitorujących
pozwalające na odbieranie danych będących wynikiem pasywnych sprawdzeń
wykonowanych u klientów mobilnych. Wskazana jest próba użycia tych
narzędzi. Możliwa jest jednak sytuacja, w której narzędza przeznaczone
dla klientów statycznych nie będą spełniały wymagań stawianych przed
tym modułem odbiorczym. Należy wówczas rozważyć modifikację
istniejącego narzędzia lub napisanie nowego od podstaw.


\section[Moduł mobilny][Moduł mobilny]{Moduł mobilny}

Moduł ten jest odpowiedzialny za monitorowanie zadanych parametrów
urządzenia mobilnego. Poniewż klienty mobilne są od siebie nie zależne
i mogą oprować bez możliwosci komunikacji ze sobą, konieczne jest aby
każdy klient mobilny posiada swoją instancję tego modułu, która jest
odpowiedzialna za monitorowanie jego urządzenia. Ten elementn systemu
musi posiadać budową modularną. Najważniejsze z wymagań
odnoszocych się do tego modułu narzuca, aby możliwe było w jak
najprostszy sposób dodawanie możliwości sprawdzania nowych parametrów.

Implementacja tego modułu musi uwzględniać uwarunkowania sprzętowe jak
i systemowe platformy na której się znajduje. Urządzenia mobilne są
zazwyczaj zasilane z~własnych akumulatorów dlatego konieczne jest
zastosowanie mechanizmów, które pozwolą na zredukowanie zużycia
energii związanego z~systematycznym wykonywaniem sprawdzeń. Należy
równiez wspomnieć, iż moduł mobilny odpowiedzialny jest za nadawianie
każdemu z~odczytów stempla czasu uniwersalnego\footnote{Czas
  uniwersalny - średni astronomiczny czas słoneczny na południku
  zerowym.} dokonywanego pomiaru. Na podstawie dokonanej
charakterystyki klienta mobilnego, można poczynić założenie, iż klient
posiada dostęp do punktów synchronizacji czasu. Jest wiele dostępnych
metod synchronizacji czasu na urządzeniu mobilnym, midzy innymi
pobranie czasu z~sieci GSM czy też z~serwerów czasu światowego, przez
co nie stanowi to dla klienta mobilnego poważnego wymagania. 

Klient mobilny po zebraniu porcji wpisów dziennika o~rozmiarze zgodnym
z~polityką administratora, lub po upływie określonego czasu powinien
przesłać posiadane wpisy dziennika do modułu odbiorczego, a~po
uzyskaniu potwierdzenia usunąć je z~urządzenia w~celu oszczędności
pamięci. Różnorodność platform dostępnych na rynku sprawia, iż nie
można wymagać od modułu odbiorczego dostarczenia uniwersalnej
implementacji protokołu komunikacyjnego. Wymaga się zatem, aby klient
mobilny używał protokołu komunikacyjnego zgodnego z protokołem modułu
odbiorczego. Konieczne jest również, aby klient mobilny posiadał
możliwość definiowania metod uwierzytelnienia. Nalezy również zapewnić
możliwość sprawdzenia autentyczności serwera, z którym nawiązuje się
połączenie.

%dodać referencje do pracy kubika
W ramach systemu monitorowanie funkcjonować będzie wiele instancji
modułu mobilnego. Instancje te mogą używać bardzo wielu platform. W
chwili pisania tej pracy nie znaleziono na rynku żadnej aplikacji
przeznaczonej, na platformę mobilną, która spełniałaby to założenie. W
omawianym systemie wykorzystano moduł mobilny, przeznaczony dla
platformy Android, który został zaprojektowany i zaimplementowany
przez Pana Marcina Kubika. Szczegółowy opis tej implementacji klienta
mobilnego można znaleźć w [praca\_kubika].
