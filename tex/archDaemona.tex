\chapter{Architektura modułu odbioru danych}
\label{chap:ArchDaemona}

\section[Analiza][Analiza]{Analiza}

Na rynku brak jest rozwiazań dedykowanych do monitoringu klienta
mobilnego. Rozwiązania przeznaczone dla klientów statycznych, takie
jak dodatek NSCA, nie spełniają bardzo wielu wymagań, przez co ich
użycie w budowanym systemie nie jest możliwe. W związku z powyższym
zaprojektowany został i zaimplementowany moduł odbiorczy, stanowiący
dodatek do systemów rodziny Nagios. Dodatek ten jest w pełni
uniwersalny, można go wykorzystać, zarówno do monitorowania pasywnego
klientów statycznych, jak i do monitorowania klientów
mobilnych. Podczas projektowania oraz implementacji, szczególny nacisk
położono na monitorowanie klienta mobilnego. Analiza wymagań
przedstawionych w \ref{chap:Wymagania}, doprowadziła do wyznaczenia
podzbioru wymagań, które powinny być spełnione przez ten moduł. Należy
zwrócić uwagę, że istnieje również znaczny podzbiór wymagań, powiązany
z modułem odbioru danych, jednak ich realizacja jest uzależniona od
wybranego protokołu komunikacyjnego, zatem one omówione w
\ref{chap:ProtKom}. W \ref{tab:RealWymOdb} przedstawiono podzbiór
wymagań, odnoszących się do modułu odbiorczego, a także sposób ich
realizacji.

\begin{longtable}[c]{|c||p{3.5cm}|p{9cm}|}
\caption[Realizacja wymagań przez moduł odbiorczy]{Realizacja wymagań przez moduł odbiorczy} \label{tab:RealWymOdb} \\ 
  \hline
  Kod & \multicolumn{1}{c|}{Nazwa} & \multicolumn{1}{c|}{Opis} \tabularnewline
  \hline \hline
  \endfirsthead

  \multicolumn{3}{c}%
  {{\tablename\ \thetable{} -- Kontynuacja z~poprzedniej strony}} \\
  \hline
  Kod & \multicolumn{1}{c|}{Nazwa} & \multicolumn{1}{c|}{Opis} \tabularnewline
  \hline \hline
  \endhead

  \hline \multicolumn{3}{|r|}{{Kontynuacja na następnej stronie}} \\ \hline
  \endfoot

  \hline\hline
  \endlastfoot

  W5 & \raggedright{Dodawanie algorytmów} & \raggedright{Możliwe jest to poprzez dostarczenie implementacji odpowiedniego interfejsu. Szerszy opis tego zagadnienia znajduje się w \ref{sec:ModCrypto}.} \tabularnewline
  \hline

%referencja do pracy kubika
  W7 & \raggedright{Wymienne algorytmy uwierzytelnienia klienta} & \raggedright{Możliwe jest definiowanie dowolnych metod autoryzacji klienta, poprzez dostarczenie implementacji odpowiedniego interfejsu. W celu komunikacji z klientem mobilnym został algorytm autoryzacji posiada dostep do bezpiecznego kanału danych. Szerszy opis tego zagadnienia znajduje się w \ref{sec:ModAuth}\footnote{Konieczne jest dostarczenie również odpowiedniej implementacji algorytmu dla klienta mobilnego. Szczegóły dla platformy Android zostały opisane w [praca\_kubika]}.} \tabularnewline
\hline

  W10 & \raggedright{Dostarczanie w wiele miejsc} & \raggedright{Moduł odbiorczy pozwala na przekazywanie danych do wielu lokalizacji i podsystemów docelowych, bez konieczności ich retransmisji. Konfiguracja reguł zapisana jest w pliku konfiguracyjnym. Szerszy opis implementacji tego mechanizmu znajduje się w \ref{sec:ModBase}.} \tabularnewline
  \hline

 W11 & \raggedright{Reguły definiowane dla każdego klienta} & \raggedright{Możliwe jest definiowanie reguł dostarczania danych od konkretnych klientów. Ponadto możliwe jest definiowanie grup klientów i reguł dla nich.} \tabularnewline
  \hline

  W12 & \raggedright{Oszczędność pasma} & \raggedright{Program stosuje wewnętrzne bufory, co umożliwia przesłanie potwierdzenia przetworzenia danych zanim jeszcze trafią one do miejsc docelowych. Szczegółowy opis tego mechanizmu znajduje się w \ref{sec:ModBase}} \tabularnewline
  \hline

  W13 & \raggedright{Integracja z~istniejącymi systemami} & \raggedright{Moduł może być wykorzystywany z wieloma istniejącymi systemami monitorowania. Dodawanie nowych sposobów przekazywania danych do miejsca doceloweg możliwe jest poprzez dostarczenie implementacji odpowiedniego interfejsu. Szerszy opis tego zagadnienia znajduje się w \ref{sec:ModBase}.} \tabularnewline
  \hline

  W16 & \raggedright{Kontrola danych wejściowych} & \raggedright{Program pozwala na definiowanie reguł, określających uprawnienia klientów do zgłaszania odczytów parametrów danego urządzenia czy też usługi. Reguły definiowane są w pliku konfiguracyjnym. Szerszy opis mechanizmu filtrowania danych znajduje się w \ref{sec:ModBase}.} \tabularnewline
  \hline

\end{longtable}


\section[Opis architektury][Opis architektury]{Opis architektury}

Wysokopoziomowa struktura logiczna programu zakłada istnienie dwóch
podstawowych obiektów. Są nimi producenci danych oraz konsumenci. Dane
od klienta mobilnego dostarczane są przez producentów danych, a więc
muszą oni zawierać implementacje protokołów komunikacyjnych oraz
wszystkiego co z tym związane. Konsumenci danych natomist zajmują się
przekazywaniem danych do zewnętrznych programów. W celu przekazania
danych od producenta danych do konsumentów, zgodnie z zasadami
trasowania danych, potrzebny jest kanał komunikacyjny. Aby użyć kanału
komunikacyjnego potrzebna jest wiadomość, a więc poprawnie
sformatowana porcja danych.

%obrazek z high level design (tak jak powyzej)

Fizyczna struktura programu została utworzona z użyciem biblioteki Qt
jako podstawowego szkieletu aplikacji. Wykorzystano równiez biblioteki
boost oraz Crypto++. Moduł ten przeznaczony jest, podobnie jak system
monitorujący Icinga dla komputerów pracujących pod kontrolą systemu
operacyjnego Linux i jest uruchamiany jako samodzielny serwis. Fizyczna struktura programu składa się z następujących elementów:

\begin{description}
\item[Szkielet programu] Zawiera on elementy programu, konieczne do
  wytworzenia środowiska dla funkcjonowania pozostałych modułów oraz
  zarządzania nimi. Ponadto zawiera implementacje źródeł, a także ujść
  danych.
\item[Moduł kryptograficzny] Dostarcza on implementacji funkcji
  kryptograficznych, wymaganych podczas komunikacji z klientem
  mobilnym. Zawiera on zarówno algorytmy asymetryczne, konieczne do
  inicjalizacji kryptografi symetrycznej, jak i algorytmy symetryczne,
  służące do przesyłania danych.
\item[Moduł autoryzacji klienta] Zawiera on implementację algorytmów
  uwierzytelnienia klienta.
\item[Moduł komunikacji z użyciem TCP] Dostarcza on implementację
  protokołu komunikacyjnego używanego do komunikacji z klientem
  opartego na protokole TCP.
\item[Moduł logowania] Pozwala on na przekazywanie wiadomości z
  dowolnych miejsc znajdujących się w innych modułach, która zawiera
  informacje o zaistaniałym błędzie, lub innym zdarzeniu wymagającym
  poinformowania użytkownika.
\end{description}

%dodac referencje do ksiazki gammy - fabryka

Odwzorowanie podstawowych elementów struktury logicznej w fizyczną
znajduje się w szkielecie aplikacji. Pozostałe elementy programu
zapewniają elastyczność przy rozbudowie aplikacji. Szczególnym
przykładem tego usługowego charakteru pozostałych modułów może być
moduł kryptograficzny i moduł autoryzacji klienta. Zostały one
zaprojektowane jako moduły dostarczające dobrze zdefiniowane usługi
usługi dla pozostałych elemetów programu. Udostępniają one generyczne
interfejsy, które są następnie w nich implementowane. Utworzenie
obiektów pochodzących z tych modułów uzależnione jest od bierzącej
konifguracji porgramu, a ich zawartość moze być bardzo dynamiczna. W
celu ograniczenia wpływu rozrostu liczby klas w tych modułach
wykorzystano wzorzec fabryki. Każdy z omawiancy modułów zawiera klasę,
w której rejestrowana jest dostępność poszczególnych
algorytmów. Pozostałe moduły uzyskują instancje tych algorytmów,
poprzez klasę fabryki. Jeśli dany algorytm jest dostępny, zostanie on
wówczas przekazany wywołującemu i będzie on mógł uzywać go poprzez
standardowy, zdefiniowany wcześniej interface. Należy wspomnieć, iż
poszczególne klasy fabryk wykorzystują wzorzec projektowy nazywany
signleton. Oznacza to iż istnieje w programie tylko i wyłącznie jedna
instancja danej fabryki, co umożliwa zachowanie spójności i
dostępności danych w całym systemie. Zastosowanie wzorca projektowego
fabryki pozwala pozostałym obiektom, na korzystanie z obiektów,
których typ faktyczny jest nieznany w trakcie kompilacji
programu. Jest to niezbędne, gdyż faktyczny typ wykorzystywanego
algorytmu kryptograficznego czy też algorytmu autoryzacji użytkownika,
określany jest na podstawie pliku konfiguracyjnego.

W celu konfuguracji programu wykorzystano zewnętrzny plik w formacie
XML. Umożliwaia to zmianę konfiguracji programu bez konieczności jego
ponownej komilacji. Plik konfuguracyjny składa się z czterech zasadniczych sekcji:

\begin{description}
\item[Sekcja dostawców danych] zawiera dane dostawców, którzy mają
  zostać uruchomieni podczas startu programu. Umożliwia przekazanie
  dodatkowych informacji do obiektu dostawcy np. adresu IP lub portu
  na którym powinien on oczekiwać na połączenia. Ponadto w tej sekcji
  definiowane są grupy dostawców.
\item[Sekcja odbiorców danych] zawiera dane odbiorców danych, którzy
  mają zostać uruchomieni podczas startu programu. Umożliwia
  przekazanie dodatkowych informacji do obiektu odbiorcy danych,
  takich jak ścieżka do pliku do którego nalezy zapisywać dane. Sekcja
  ta umożliwia również definiowanie grup odbiorców.
\item[Sekcja definicji klientów] zawiera definicję klientów oraz grup
  klientów. Każda definicja klienta składa się z następujących sekcji:
  \begin{description}
  \item[Sekcja autoryzacji] zawiera dane o dozwolonych modułach
    autoryzacyjnych dla danego klienta. Umożliwia także dodatkową
    konfigurację instancji modułów przeznaczonych dla danego klienta.
  \item[Sekcja filtrowania] zawiera urządzenia oraz usługi, których
    dane monitorowania mogą być przesyłane przez tego konkretnego
    klienta.
  \end{description}
\item[Sekcja definicji ścieżek danych] zawiera definicję ścieżek
  danych w programie. Pozwala na definiowanie, do którego odbiorcy
  danych mają trafić dane odebrane od wskazanego klienta.
\end{description}

Podczas uruchamiania programu, plik konfiguracyjny zostaje przeczytany
oraz sprawdzony pod kątem poprawności, zarówno składniowej jak i
syntaktycznej. Niestety, ponieważ obiekty odbiekty dostawców oraz
odbiorców danych są dostarczane z zewnętrz prawidłowość ich ustawień
nie moze być sprawdzana na tym samym etapie. Jest to wykonywane
dopiero w trakcie inicjalizacji danego obiektu. Należy zatem zawsze po
pomyślnej analizie pliku konfiguracyjnego sprawdzić zawartość pliku
dziennika wykonania programu.

\section[Szkielet programu][Szkielet programu]{Szkielet programu}
\label{sec:ModBase}

Moduł ten zawiera podstawowe komponenty programu. Zawarto tutaj
wszystkie czynności przygotowawcze związane z wczytaniem konfiguracji
oraz powołaniem do życia obiektów wymaganych przez nią. Ponadto w
module tym zawarto definicję podstawowych bytów logicznych programu. W
zależności od pełnionej funkcji można wyróżnić następujące grupy obiektów:

\begin{description}
\item[Grupa obiektów konfiguracyjnych] zawiera wszystkie obiekty
  używane zarówno do wczytania parametrów uruchomienia programu z
  linii poleceń, jak również obiekty odpowiedzialne za dostarczenie do
  programu konfiguracji zawartej w pliku.
\item[Grupa obiektów zarządzających] zawiera zarządce programu oraz
  obiekty pomocnicze. Wykonywane są tutaj wszelkie czynności, które
  nalezy wykonać w trakcie uruchamiania programu, oraz powoływanie
  oraz niszczenie obiektów odwzorowań główych obiektów logicznych
  programu.
\item[Grupa obiektów producentów danych] zawiera generyczny interfejs
  producenta danych, fabrykę, umożliwającą uzyskiwanie obiektów z tej
  grupy oraz definicję dostępnych producentów danych.
\item[Grupa obiektów konsumentów danych] zawiera generyczny interfejs
  konsumenta danych, fabrykę, umożliwającą uzyskiwanie obiektów z tej
  grupy oraz definicję dostępnych konsumentów danych.
\item[Grupa obiektów kanału komunikacyjnego] zawiera obiekty powiązane
  z kanałem komunikacyjnym pomiędzy producentami danych, a ich
  konsumentami. Zawiera również mechanizmy formatowania danych oraz
  bufory przeznaczone na dane oczekujace na przekazanie.
\end{description}

\section[Moduł kryptograficzny][Moduł kryptograficzny]{Moduł kryptograficzny}
\label{sec:ModCrypto}

\section[Moduł autoryzacyjny][Moduł autoryzacji klienta]{Moduł autoryzacji klienta}
\label{sec:ModAuth}

\section[Moduł TCP][Moduł komunikacji z wykorzystaniem TCP]{Moduł komunikacji z wykorzystaniem TCP}

\section[Moduł logowania][Moduł logowania]{Moduł logowania}
