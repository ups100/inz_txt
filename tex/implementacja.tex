\chapter{Implementacja}
\label{chap:Implementacja}

\section[Opis architektury][Opis architektury]{Opis architektury}

Logiczna architektura programu została przedstawiona
w~\ref{sec:ProjModOdb}. Fizyczna struktura programu została utworzona
z~użyciem biblioteki Qt jako podstawowego szkieletu
aplikacji. Wykorzystano również biblioteki boost oraz Crypto++. Moduł
ten przeznaczony jest, podobnie jak system monitorujący Icinga dla
komputerów pracujących pod kontrolą systemu operacyjnego Linux i~jest
uruchamiany jako samodzielny serwis. Fizyczna struktura programu
składa się z~następujących elementów:

\begin{description}
\item[Szkielet programu] Zawiera on elementy programu, konieczne do
  wytworzenia środowiska dla funkcjonowania pozostałych modułów oraz
  zarządzania nimi. Ponadto zawiera implementacje dostawców oraz
  konsumentów danych.
\item[Moduł kryptograficzny] Dostarcza on implementacji funkcji
  kryptograficznych, wymaganych podczas komunikacji z klientem
  mobilnym. Zawiera on zarówno algorytmy asymetryczne, konieczne do
  inicjalizacji kryptografii symetrycznej, jak i~algorytmy symetryczne,
  służące do przesyłania danych.
\item[Moduł autoryzacji klienta] Zawiera on implementację algorytmów
  uwierzytelnienia klienta.
\item[Moduł komunikacji z użyciem TCP] Dostarcza on implementację
  protokołu komunikacyjnego używanego do komunikacji z~klientem.
\item[Moduł logowania] Pozwala na przekazywanie użytkownikowi
  komunikatów z~dowolnych miejsc znajdujących się w~innych
  modułach. Wiadomość ta może zawierać informacje o~zaistniałym
  błędzie, lub innym zdarzeniu wymagającym poinformowania użytkownika.
\end{description}

%rysunek o wspolpracy tych elementow

Odwzorowanie podstawowych elementów struktury logicznej w~fizyczną ma
miejsce w~szkielecie aplikacji. Pozostałe elementy programu zostały
zaprojektowane jako moduły pomocnicze świadczące dobrze zdefiniowane
usługi. Szczególnym przykładem, tego usługowego charakteru pozostałych
modułów, może być moduł kryptograficzny i~moduł autoryzacji
klienta. Udostępniają one generyczne interfejsy dostępu do swoich
usług dla pozostałych modułów. Szczegóły implementacyjne są natomiast
zamknięte wewnątrz modułów. Typ faktyczny obiektu udostępnianego
poprzez generyczny interfejs jest w~bardzo wielu przypadkach
determinowany dopiero na etapie konfiguracji programu. Ponadto liczba
klas znajdujących się w~tych modułach może znacząco rosnąć. Jednym z
wymagań było zapewnienie możliwości definiowania nowych algorytmów
kryptograficznych jak i~algorytmów uwierzytelnienia klienta. W~celu
zapewnienia możliwości wybrania typu faktycznego obiektu na podstawie
danych wykorzystano wzorzec projektowy fabryki\footnote{Wzorzec ten
  został szczegółowo opisany w XXX}.  Każdy z~omawianych modułów
posiada swojego zarządcę. Dostępne w~module obiekty muszą zostać
zarejestrowane u~swojego zarządcy. Pozostałe moduły uzyskują instancje
tych obiektów, poprzez zarządcę, który na podstawie przekazanych
danych określa typ faktyczny obiektu. Jeśli dany typ jest dostępny,
zostanie on wówczas przekazany wywołującemu i~będzie on mógł używać go
poprzez dostarczony generyczny interfejs. Logiczna struktura tych
modułów wymaga zapewnienia, że w~programie istnieje tylko jedna
instancja obiektu danego zarządcy. W~tym celu został wykorzystany
wzorzec projektowy nazwie singleton. Zastosowanie wzorca projektowego
fabryki pozwala pozostałym modułom, na korzystanie z obiektów, których
typ faktyczny jest nieznany w~trakcie implementacji oraz kompilacji
programu.

W celu konfiguracji programu wykorzystano zewnętrzny plik w formacie
XML. Umożliwia to zmianę ustawień programu bez konieczności jego
ponownej kompilacji. Plik konfiguracyjny składa się z~czterech
zasadniczych sekcji:

\begin{description}
\item[Sekcja dostawców danych] zawiera dane dostawców, którzy mają
  zostać uruchomieni podczas startu programu. Umożliwia przekazanie
  dodatkowych informacji do obiektu dostawcy np. adresu IP lub portu
  na którym powinien on oczekiwać na połączenia. 
\item[Sekcja odbiorców danych] zawiera dane odbiorców danych, którzy
  mają zostać uruchomieni podczas startu programu. Umożliwia
  przekazanie dodatkowych informacji do obiektu odbiorcy danych,
  takich jak ścieżka do pliku do którego należy zapisywać dane. 
\item[Sekcja definicji klientów] zawiera definicję klientów oraz grup
  klientów. Każda definicja klienta składa się z następujących sekcji:
  \begin{description}
  \item[Sekcja autoryzacji] zawiera dane o dozwolonych modułach
    autoryzacyjnych dla danego klienta. Umożliwia także dodatkową
    konfigurację instancji modułów przeznaczonych dla danego klienta.
  \item[Sekcja filtrowania] zawiera urządzenia oraz usługi, których
    dane monitorowania mogą być przesyłane przez tego konkretnego
    klienta.
  \end{description}
\item[Sekcja definicji ścieżek danych] zawiera definicję ścieżek
  danych w~programie. Pozwala na definiowanie, do którego odbiorcy
  danych mają trafić dane odebrane od wskazanego klienta.
\end{description}

Podczas uruchamiania programu, plik konfiguracyjny zostaje przeczytany
oraz sprawdzony pod kątem poprawności zarówno składniowej jak
i~semantycznej. Obiekty obiekty obiekty dostawców, odbiorców oraz
algorytmów uwierzytelnienia dostarczane są przez zewnętrznych
programistów, zatem prawidłowość ich ustawień nie może być sprawdzana
na tym samym etapie. Jest to wykonywane dopiero w~trakcie
inicjalizacji danego obiektu. Należy zatem zawsze po pomyślnej
analizie pliku konfiguracyjnego sprawdzić zawartość pliku dziennika
wykonania programu.

\section[Szkielet programu][Szkielet programu]{Szkielet programu}

Moduł ten zawiera podstawowe komponenty programu. Zawarto tutaj
wszystkie czynności przygotowawcze związane z~wczytaniem konfiguracji
oraz utworzeniem obiektów wynikających z~niej. Ponadto w~module tym
znajdują się definicję podstawowych bytów logicznych
programu. W~zależności od pełnionej funkcji można wyróżnić następujące
grupy obiektów:

\begin{description}
\item[Grupa obiektów konfiguracyjnych] zawiera wszystkie obiekty
  używane zarówno do wczytania parametrów uruchomienia programu z~
  linii poleceń, jak również obiekty odpowiedzialne za dostarczenie do
  programu konfiguracji zawartej w pliku.
\item[Grupa obiektów producentów danych] \raggedright{zawiera generyczny interfejs
  producenta danych oraz fabrykę, umożliwiającą pozyskiwanie obiektów
  z~tej grupy, a~także definicję dostępnych producentów danych.}
\item[Grupa obiektów konsumentów danych] zawiera generyczny interfejs
  konsumenta danych oraz fabrykę, umożliwiającą pozyskiwanie obiektów
  z~tej grupy, a~także definicję dostępnych konsumentów danych.
\item[Grupa obiektów filtrujących] zawiera obiekty pozwalające na
  kontrolę danych otrzymywanych od klienta
\item[Grupa obiektów kanału komunikacyjnego] zawiera obiekty powiązane
  z~kanałem komunikacyjnym pomiędzy producentami danych, a~ich
  konsumentami. Zawiera również mechanizmy formatowania danych oraz
  bufory przeznaczone na dane oczekujące na przekazanie.
\item[Grupa obiektów zarządzających] zawiera zarządce programu oraz
  obiekty pomocnicze. Wykonywane są tutaj wszelkie czynności, które
  należy wykonać w~trakcie uruchamiania programu, a~także tworzenie
  oraz niszczenie obiektów implementujących elementy logiczne
  programu.
\end{description}

Głównym członkiem grupy obiektów konfiguracyjnych jest parser pliku
konfiguracyjnego. Ponieważ plik konfiguracyjny posiada strukturę pliku
XML możliwe było wykorzystanie czytelnika strumienia XML z~biblioteki
Qt. Klasa ta zapewnia generację znaczników oraz sprawdzanie
poprawności składniowej czytanego dokumentu. Umożliwiło to
implementację prostego parsera rekursywnie zstępującego, który zajmuje
się jedynie sprawdzaniem poprawności logicznej znaczników. Ze względu
na strukturę plików XML, nie jest możliwa pełna bieżąca kontrola
danych w~nim zawartych. Konieczne jest zatem wczytanie pliku
konfiguracyjnego i~odwzorowanie go w~strukturach danych, a~następnie
wykonanie sprawdzenia spójności oraz poprawności tych danych. Obiekt
parsera konfiguracji jest również globalnym obiektem udostępniającym
parametry konfiguracji. W~obiekcie tym znajdują się wszystkie
ustawienia oraz definicje wszystkich obiektów logicznych programu.

Grupa obiektów producentów danych składa się z~dwóch głównych
elementów. Pierwszym z~nich jest klasa implementująca wzorzec
fabryki. Pozwala ona pozostałym obiektom na uzyskiwanie instancji
obiektu dostawcy danych, bez konieczności znania jego typu faktycznego
w~trakcie pisania kodu czy też kompilacji. Drugim z~elementów jest
generyczny interfejs dostawcy danych, który musi być implementowany
przez każdego dostawcę. Interfejs ten pozwala na wykonywanie
wszystkich niezbędnych operacji. Między innymi na przekazanie
konkretnej instancji klasy dodatkowych danych inicjujących takich jak
adres sieciowy czy numer portu. Należy również nadmienić, że konieczne
było również dostarczenie odpowiedniego mechanizmu rejestracji
definiowanych obiektów w~fabryce. Istotne jest, aby umożliwić
programiście rejestrację nowego typu faktycznego obiektu bez
konieczności ingerencji w~inne pliki źródłowe. W~celu ułatwienia tego
procesu zostało opracowane makro, które dzięki wykorzystaniu szablonów
dokonuje automatycznej rejestracji nowego typu faktycznego obiektu
w~fabryce. Dzięki jego wykorzystaniu programista, podczas dodawania
nowego dostawcy danych musi zapewnić jedynie deklarację oraz definicję
nowego typu. Warto zauważyć, że sposób implementacji tego makra
pozwala na umieszczenie definicji nowego typu w~pliku nagłówkowym,
który jest włączany do wielu jednostek translacji i nie powoduje to
błędów kompilacji ani błędów funkcjonowania procesu rejestracji.

\vspace{0.5cm}
\begin{minipage}{\textwidth}
\begin{lstlisting}[language=c++, caption=Definicja dostawcy danych, linewidth=10cm]
DATA_PROVIDER(NazwaTypu, NazwaRejestrowana)
{
  //deklaracja metod
}
\end{lstlisting}
\end{minipage}
\vspace{0.5cm}

W~ramach tej grupy obiektów dostarczono również referencyjną
implementację dostawcy o~nazwie typu DefaultTcpProvider. Stanowi on
implementację omówionego wcześniej protokółu komunikacyjnego. W~celu
zapewnienia lepszej wydajności, logika funkcjonowania tego dostawcy
została przeniesiona do osobnego wątku programu. Aby zapewnić
elastyczności wykorzystania tego obiektu możliwe jest definiowanie
z~poziomu pliku konfiguracyjnego zarówno adresu IP i~portu
wykorzystywanego przez ten obiekt, a~także wskazanie pliku
zawierającego klucz prywatny.

Grupa obiektów konsumentów danych posiada analogiczną budowę jak grupa
producentów danych. Zapewnia ona zarówno fabrykę konsumentów danych
jak i~generyczny interfejs konsumenta. Rejestracja obiektów w~fabryce
odbywa się również przy użyciu analogicznego makra. W~ramach tej grupy
obiektów dostarczono implementację dwóch konsumentów danych. Pierwszy
z~nich o~nazwie typu ToScreenPrinter, spełnia jedynie funkcję
kontrolną. Wszystkie dane, które do niego trafią są natychmiast
wypisywane w~dzienniku wykonania programu. Dostawca typu
ToIcingaWriter odpowiedzialny jest natomiast za przekazywanie
wszystkich danych do pliku komend zewnętrznych systemu Icinga. Możliwa
jest zmiana ścieżki pliku komend zewnętrznych systemu Icinga poprzez
plik konfiguracyjny.

Grupa obiektów filtracji pozwala na kontrolę danych otrzymywanych od
klienta. Każdy klient posiada zdefiniowany w~pliku konfiguracyjnym
zbiór urządzeń i~usług o~których informacje może przesyłać. Obiekty
z~tej grupy pobierają z~analizatora składniowego wspomniane zbiory
i~dokonują kontroli każdego wpisu dziennika otrzymanego od
klienta. Jeśli klienta nadesłał dane dotyczące urządzenia lub usługi
do których nie ma on uprawnień, nie będzie możliwe przekazanie ich do
konsumentów.

Grupa obiektów kanału komunikacyjnego odpowiedzialna jest za
niezawodne przekazanie danych od dostawcy danych do konsumentów według
reguł zdefiniowanych w~pliku konfiguracyjnym. Zapewnienie
niezawodności zostało osiągnięte poprzez implementację bufora kołowego
wewnątrz pliku. Każdy dostawca danych posiada swój plik bufora. Na
początku tego pliku zapisane są położenia miejsca przeznaczonego do
czytania oraz miejsca przeznaczonego do pisania. Dane, które logicznie
znajdują się pomiędzy miejscem do czytania, a~miejscem do pisania są
to dane, które zostały odebrane od klienta lecz nie zostały jeszcze
dostarczone do konsumentów. Pomyślnie zweryfikowana przez obiekty
filtrujące porcja danych przekazywana jest z~użyciem kanału
komunikacyjnego do konsumentów danych. Operacja ta odbywa się w~dwóch
etapach. Pierwszy etap dokonuje zapisu danych do pliku bufora. Jeśli
aktualnie nie ma żadnych danych oczekujących na przetworzenie przez
konsumentów, nowa porcja danych jest niezwłocznie dostarczana do
odpowiednich obiektów. Jeśli istnieją porcje danych, które oczekują na
zapisanie dane zostaną zapisane do pliku i dostarczone, po danych,
które nadeszły przed nimi. Należy zwrócić uwagę, że dane zapisywane są
na dysku tylko raz, niezależnie od liczby dostawców do których powinny
one zostać dostarczone.

Głównym przedstawicielem grupy obiektów zarządzających jest klasa główna
programu. Program został napisany zgodnie z~metodyką obiektową zatem
cała logika wykonania programu została również zamknięta w~klasie co
uprościło funkcję główną programu do minimum. Klasa ta odpowiedzialna
jest za przebieg całości programu. Pierwszą operacją wykonywaną przez
tą klasę jest odnalezienie pliku konfiguracyjnego i~zlecenie jego
wczytania przez parser konfiguracji. Na podstawie informacji
uzyskanych w~wyniku analizy pliku konfiguracyjnego klasa główna
programu pobiera z~odpowiednich fabryk wszystkie obiekty producentów
oraz konsumentów zdefiniowanych w~pliku konfiguracyjnym oraz
przekazuje im odpowiednie parametry inicjalizacyjne. Klasa ta jest
również odpowiedzialna za prawidłową deinicjalizację oraz destrukcję
wszystkich obiektów. Ponadto należy zauważyć, że omawiany program
funkcjonuje jako serwis systemowy, zatem konieczne jest również
wykonanie w~tej klasie wszystkich czynności zalecanych przy
uruchamianiu takich serwisów.

\section[Moduł kryptograficzny][Moduł kryptograficzny]{Moduł kryptograficzny}

Moduł ten dostarcza pozostałym elementom programu implementacji
algorytmów kryptograficznych. Dostępne są implementacje następujących
schematów algorytmów:

\begin{itemize}
\item szyfrowanie symetrycznych,
\item szyfrowanie asymetrycznych,
\item funkcja skrótu,
\item podpis cyfrowy.
\end{itemize}

Każdy ze schematów posiada zdefiniowany generyczny
interfejs. Dostarczanie implementacji danego schematu
kryptograficznego odbywa się w~sposób analogiczny do dostarczania
implementacji dostawców danych. Wszystkie implementacje algorytmów
zostają zarejestrowane w~fabryce kryptograficznej, która umożliwia
uzyskiwanie obiektu o~typie określanym na podstawie danych na przykład
odebranych od klienta..

% biblio do CBC i AESA
W~ramach tej pracy dostarczono kilku algorytmów, które były konieczne
do zaimplementowania protokołu komunikacyjnego. Jako algorytm
symetryczny dostarczona została implementacja algorytmu AES
pracującego w trybie wiązania bloków zaszyfrowanych. Protokół
komunikacyjny wymagał dostarczenia również asymetrycznego algorytmu
RSA. W module zdefiniowano ponadto klasę implementującą generyczny
interfejs funkcji skrótu, która dostarcza funkcjonalności algorytmu
SHA-2 o długości skrótu 256 bitów. Jeden z etapów protokołu
komunikacyjnego wymagał również dostarczenia algorytmu podpisu
cyfrowego opartego na algorytmie RSA.

Do implementacji wszystkich algorytmów została wykorzystana biblioteka
Crypto++. Jest to popularna biblioteka o otwartych źródłach napisana w
języku C++, która w obiektowy sposób udostępnia algorytmy
kryptograficzne.

\section[Moduł uwierzytelnienia][Moduł uwierzytelnienia klienta]{Moduł uwierzytelnienia klienta}

Moduł posiada architekturę typową dla modułów usługowych. Głównym
elementem jest klasa implementująca wzorzec fabryki oraz generyczny
interfejs pozwalający na wykorzystywanie obiektów uzyskanych
z~fabryki.

Interfejs zdefiniowany dla algorytmów uwierzytelnienia został
zaprojektowany z~wykorzystaniem mechanizmu sygnałów i~slotów
z~biblioteki Qt. Użycie tego mechanizmu pozwala na znacznie bardziej
wydajne wykorzystanie zasobów. Przykładem takiej optymalizacji może być
czas gdy klient przetwarza żądanie związanie z uwierzytelnieniem wątek
serwera nie musi być wtedy bezczynny lecz może przetwarzać żądania
pochodzące od innych klientów. Definicja interfejsu pozwala również na
przekazanie do niego dodatkowych ustawień pochodzących z~pliku
konfiguracyjnego. Każdy algorytm uwierzytelnienia powinien po
zakończeniu sukcesem lub porażką procesu uwierzytelnienia klienta
wykonać emisję sygnału z~interfejsu algorytmu wraz z~rezultatem
procesu autoryzacji.

Zapewnienie niezależności implementacji algorytmów uwierzytelnienia od
wykorzystywanej aktualnie metody komunikacji wymagało zdefiniowania
generycznego interfejsu komunikacyjnego, który może być wykorzystywany
przez implementacje poszczególnych algorytmów. Implementacja tego
interfejsu powinna być dostarczona przez moduł, który jest aktualnie
wykorzystywany w~programie do komunikacji z~klientem.

W~ramach pracy zostały również zaimplementowane dwa moduły
uwierzytelnienia klienta. Pierwszy z nich nosi nazwę AlwaysAllow
i~jest to tak zwane uwierzytelnienie puste, czyli zakończone sukcesem
dla każdego klienta. Drugi moduł - LoginPass jest to prosta metoda
uwierzytelnienia oparta na pobraniu od klienta loginu oraz hasła
i~porównanie go z~danymi dostarczonymi w~pliku konfiguracyjnym. Każdy
klient posiada w~pliku konfiguracyjnym listę dostępnych dla niego
algorytmów uwierzytelnienia wraz z~danymi jakie powinny być
przekazane, aby zapewnić pozytywne wykonanie procesu. Należy zwrócić
uwagę, że zaimplementowane algorytmy uwierzytelnienia stanową jedynie
przykład i~nie powinny być wykorzystywane w~systemie produkcyjnym,
ponieważ wszystkie hasła oraz nazwy użytkowników przechowywane są
jawnym tekstem w pliku konfiguracyjnym.

\section[Moduł TCP][Moduł komunikacji z wykorzystaniem TCP]{Moduł komunikacji z wykorzystaniem TCP}

Moduł ten zawiera implementację protokołu komunikacyjnego opisanego
w~\ref{sec:ProtKom}. Inicjacja każdej z~warstw protokołu została
zaimplementowana dedykowanej klasie lub jeśli proces inicjacji złożony
był z~kilku rozdzielnych logicznie elementów, każdy element został
zaimplementowany w~osobnej klasie. W~celu umożliwienia każdej z~warstw
protokołu korzystanie z~usług warstw niższych w~sposób generyczny
wykorzystany został wzorzec dekoratora.

Aby wykorzystać wzorzec dekoratora zdefiniowano generyczny interfejs
pozwalający na odczytanie oraz zapisanie komunikatu niezależnie od
liczby warstw znajdujących się poniżej. Klasą prostą w tym przypadku
jest klasa zawierająca implementację warstwy formowania
wiadomości. Klasa ta została oparta na implementacji gniazd TCP
pochodzącej z~biblioteki Qt. Klasami dekorującymi natomiast są klasy
zapewniające szyfrowanie oraz dołączanie skrótu wiadomości.

%diagram z tymi klasami

Wykorzystanie wzorca dekoratora pozwoli w~przyszłości na łatwą
modyfikację protokołu komunikacyjnego np. poprzez dodanie dodatkowej
warstwy. Ponadto wprowadzenie jednolitego interfejsu pozwoliło na
zachowanie prostoty i~jednolitości implementacji poszczególnych warstw
protokołu komunikacyjnego.

Moduł ten został zaimplementowaniu przy użyciu licznych mechanizmów
z~biblioteki Qt. Przede wszystkim wykorzystany został moduł sieciowy
wspomnianej biblioteki. Dzięki jego użyciu uzyskano dostęp do
generycznej implementacji serwera TCP, a~także gniazd. Szkielet
aplikacji Qt pozwolił na wygodną implementację asynchronicznej
komunikacji z~użyciem gniazd TCP przy pomocy standardowego dla tego
szkieletu mechanizmu sygnałów i~slotów. Dzięki temu uzyskano
przejrzysty i~wydajny kod, który pozwala na obsługę wielu klientów
w~jednym wątku.

\section[Moduł logowania][Moduł logowania]{Moduł logowania}

Omawiany program wykonywany jest bez interakcji
z~użytkownikiem. Funkcjonuje on jako serwis systemowy. Docelowo będzie
on wykonywany na serwerze, poza sesją jakiegokolwiek
użytkownika. W~trakcie wykonania programu mogą się zdarzyć sytuacje
wymagające poinformowania użytkownika o~ich wystąpieniu. Znaczna część
z~tych informacji stanowi jedynie zapis wykonania programu, jednak
mogą występować również informacje o sytuacjach krytycznych, o których
użytkownik musi zostać powiadomiony. Konieczne było zatem dostarczenie
możliwości przekazywania takich informacji z~wielu modułów do jednego,
wspólnego miejsca, które stanowi dziennik wykonania programu.

Każdy moduł posiada możliwość przekazywania użytkownikowi wiadomości
o~różnym priorytecie. Dozwolone są następujące priorytety:

\begin{description}
\item[FATAL] najwyższy priorytet, wiadomość zawiera komunikat
  o~błędzie, który uniemożliwia dalsze wykonanie programu
\item[ERROR] komunikat zawiera informacje o~błędzie, który
  uniemożliwia wykonanie pewnej ścieżki programu
\item[WARNING] komunikat zawiera ostrzeżenie o~nietypowej sytuacji
\item[DEBUG] komunikat zawiera treść pomocną podczas wyszukiwania błędów
\item[INFO] komunikat zawiera jedynie treści informacyjne
\end{description}

Podczas kompilacji ustalany jest minimalny priorytet wiadomości, które
mają być przekazywane użytkownikowi. Wszystkie wiadomości
o~priorytecie niższym niż ustalony, nie zostaną zapisane. Ponadto
dzięki użyciu mechanizmów opartych o~szablony wszystkie komunikaty
o~priorytecie niższym zostaną rozwinięte do wywołania funkcji
pustej. Wywołanie takie zostanie z bardzo dużym prawdopodobieństwem
zoptymalizowane przez kompilator.

Przekazanie użytkownikowi treści komunikatu w~wielu sytuacjach może
nieść zbyt mało informacji. W~celu umożliwienia przekazania
dodatkowych informacji bez konieczności pisania nadmiernej liczby
komend przy każdym komunikacie, opracowana została makrodefinicja,
która do każdego komunikatu dołączy aktualny stempel czasu, nazwę
pliku w~którym znajduje się komunikat, a~także nazwę funkcji oraz
numer linii. Ponadto komunikat nie musi się składać jedynie z tekstu
lecz można go formować w~taki sam sposób jak pisać do strumienia.

\vspace{0.5cm}
\begin{minipage}{\textwidth}
  \begin{lstlisting}[language=c++, breaklines=true,
    linewidth=0.99\textwidth, caption=Przykładowe wypisanie
    komunikatu]

LOG_ENTRY(MyLogger::DEBUG, "komunikat"<<123);

\end{lstlisting}
\end{minipage}
\vspace{0.5cm}

\vspace{0.5cm}
\begin{minipage}{\textwidth}
\begin{lstlisting}[linewidth=0.99\textwidth, caption=Format komunikatu przekazywanego użytkownikowi]

[stempel czasu][poziom][plik][funkcja][linia]:komunikat123

\end{lstlisting}
\end{minipage}
\vspace{0.5cm}

Ponieważ program nie jest przypisany do żadnego z~wirtualnych
terminali nie ma możliwości przekazywania wiadomości na standardowe
wyjście lub wyjście błędów. Konieczne jest zatem utworzenie pliku, do
którego zapisywane będą komunikaty. Należy zwrócić uwagę, że program
jako serwis systemowy uruchomiony będzie ze znacznie ograniczonymi
prawami, aby podnieść poziom bezpieczeństwa serwera. W~związku
z~powyższym jedynym miejscem, co do którego można założyć, że program
będzie miał dostęp jest katalog plików tymczasowych. Każde
uruchomienie programu powoduje zatem utworzenie w~tym katalogu pliku
składającego się z~nazwy programu oraz stempla czasu zawierającego
czas jego uruchomienia.
