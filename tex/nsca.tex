\chapter{Monitorowanie rozproszone z użyciem NSCA}
\label{chap:nsca}

\section[Opis dodatku NSCA][Opis dodatku NSCA]{Opis dodatku NSCA}

NSCA - Nagios Service Check Acceptor jest to dodatek do systemów
monitorujących opartych o~system Nagios pozwalający na wykorzystanie
mechanizmów pasywnego monitorowania z~systemu innego niż ten na którym
uruchomione jest oprogramowanie monitorujące. Program ten został
napisany w całości w języku~C i~wydany na licencji GPL. Wykorzystuje
on plik zewnętrznych komend i nie integruje się z jądrem
monitorującym. Dzięki temu możliwe jest jego wykorzystanie zarówno w
systemie Nagios jak i jego klonach takich jak wspominany system
Icinga. Dodatek ten składa się z~dwóch modułów:

\begin{itemize}
\item moduł wysyłający (send\_nsca) służący do wysyłania wyników
  sprawdzeń z~monitorującego systemu do centralnego serwera, na którym
  umieszczone jest jądro odpowiedzialne za przetwarzanie wyników
  sprawdzeń,
\item moduł odbierający (nsca) służący do odbierania wyników sprawdzeń
  od klientów i~dostarczaniu ich do jądra, które zajmuje się dalszym
  ich przetwarzaniem.
\end{itemize}

%tutaj wstawic grafike z tym jak to dziala
%źrodło: dokumentacja w repo

\subsection[Moduł wysyłający][Moduł wysyłający]{Moduł wysyłający}
\label{subsec:modulWysylajacy}

Ta część dodatku uruchamiana jest na systemie, na którym funkcjonuje
jakiś mechanizm sprawdzający, który generuje wpisy dziennika. Wpisy te
po utworzeniu, przekazywane są do programu wysyłającego. Moduł
wysyłający, po uruchomieniu odczytuje ustawienia z pliku
konfiguracyjnego, a następnie próbuje połączyć się z serwerem. Po udanym
połączeniu otrzymuje pakiet inicjujący, który zawiera:

\begin{itemize}
\item wektor inicjalizacyjny: używany do celów kryptogrficznych,
  wygenerowany przez serwer pseudolosowy ciąg znaków, konieczny do
  inicjalizacji algorytmu kryptograficznego,
\item stempel czasu: czas odczytany przez serwer przez serwer w~chwili
  nadejścia połączenia od klienta.
\end{itemize} 

Po otrzymaniu pakietu inicjującego moduł rozpoczyna czytanie wpisów z
standardowego wejścia. Wszystkie wpisy dziennika muszą być odpowiednio
sformatowane. Poszczególne pola informacyjne muszą być rozdzielone
pojedyńczą tablulacją, a~cały wpis zakończony znakiem nowej
linii. Wpisy dotyczącego urządzenia powinny zawierać następujące pola:

\begin{itemize}
\item nazwa urządzenia: krótka nazwa urządzenia, którego stan jest
  przekazywany,
\item stan: numerycznie wyrażony kod stanu urządzenia,
\item odczyt: dodatkowe wartości odczytów opisujące stan urządzenia.
\end{itemize}

Natomast wpisy dotyczące konkretnej usługi tego urządzenia powinny
zawierać następujące pola:

\begin{itemize}
\item nazwa urządzenia: krótka nazwa urządzenia na którym uruchomiona
  jest usługa,
\item opis usługi: nazwa usługi danego urządzenia, której dotyczy wpis
\item stan: numerycznie wyrażony kod stanu usługi,
\item odczyt: dodatkowe wartości odczytów opisujące stan usługi.
\end{itemize}

%dodac odniesienie do odpowiedniego rozdzialu
Łatwo zauważyć, że żadne z pól wpisu w dzienniku nie zawera stempla
czasu wymaganego przez jądro sprwdzające przy zapamiętywaniu odczytu
pasywnego. Dzieje się tak, gdyż program NSCA posiada zdefiniowaną
własną politykę określania czasu wpisu w dzienniku. Do każdego pakietu
zawierającego wpis dziennika dodawany jest stempel czasu otrzymany w
pakiecie inicjującym od modułu odbierającego. Właściwy stepel czasu,
który trafia do jadra sprawdzającego nadawany jest natomiast przez
moduł odbierający.

Koejnym krokiem działania modułu jest obliczenie cyklicznego kodu
nadmiarowego CRC32 dla danego pakietu. Po dołączeniu obliczonego kodu
do pakietu pakiet jest szyfrowany. Algorytm szyfrujący stosowany do
szyfrowania pakietów został wcześniej zainicjalizowany wektorem
pseudolosowych danych odebranych w pakiecie inicjalizacyjnym od modułu
odbierającego. Po zaszyfrowaniu dane są wysyłane, a moduł wysyłający,
bez oczekiwania na potwierdzenie przetworzenia przez serwer,
rozpoczyna przetwarzanie kolejnego wpisu dziennika.

\subsection[Moduł odbierający][Moduł odbierający]{Moduł odbierający}

%dodać odniesienie do rozdzialu o pasywnym monitorowaniu
Demon, który stanowi moduł odbierający funkcjonuje na tym samym
systemie operacyjny na którym znajduje się jądro systemu
monitorującego. Ta część odpowiedzialna jest za odbieranie danych od
klientów i przekazywanie ich do jądra programu monitorującego. Moduł
ten może pracować w jednym z poniższych trybów:

\begin{itemize}
\item samodzielny demon jednoprocesowy: uruchomiony w tle demon, który
  nasłuchuje na przychodzące połączenia od klientów i po nadejściu
  połączenia jest ono obsługiwane przy użyciu jednego procesu z jednym
  wątkiem,
\item samodzielny demon wieloprocesowy: uruchomiony w tle demon,
  którego proces główny nasłuchuje na nadejście połączeń od klientów,
  gdy takie połączenie nadejdzie proces jest duplikowany i każdy z
  klientów obsługiwany jest w innym procesie potomnym,
\item demon zintegrowany z inetd: w systemie uruchomiony jest demon
  inetd, który nasłuchuje na połączenia od klientów na konkrentym
  gnieździe, a gdy nadejdzie połączenie od klienta uruchamiany jest
  proces demona NSCA, który obsługuje nowe połączenie i kończy się
  wraz z zakończeniem obsługi klienta
\end{itemize}

Do przekazywania wpisów dziennika używany jest mechanizm pasywnego
monitorowania dostępny w systemach z rodziny Nagios. Aby możliwe było
wykorzysanie tego mechanizmu konieczne jest zapewnienie demonowi
dostępu do pliku zewnętrznych komend systemu monitorującego. Ponieważ
plik zewnętrznych komend jest potokiem nazwanym, chroniony jest on
przez Uniksowy system uprawnień użytkowników. Zapewnienie dostępu do
takiego bytu może się odbyć na dwa sposoby. Pierwszym, polecanym przez
twórców systemów monitorujących, jest uruchamianie demona NSCA jako
procesu tego samego użytkownika co proces jądra systemu
monitorującego. Drugim sposobem jest modyfikacja praw dostępu do
omawianego pliku, tak aby umożliwić dostęp użytkownikowi z którego
uprawnieniami uruchomiony jest demon NSCA. Przy zastosowaniu drugiego
rozwiązania zalecana jest szczególna ostrożność, gdyż dostęp do pliku
zewnętrznych komend daje bardzo duże możliwości ingerencji w system
monitorujący.

Komunikacja modułu odbirającego z~klientem rozpoczyna się od nadejścia
połączenia od klienta. Gdy moduł odbierający otrzyma nowe połączenie
zostanie wysłany pakiet inicjalizujący, którego zawartość została
opisana w~\ref{subsec:modulWysylajacy}. Po przesłaniu pakietu
inicjalizującego połączenie, moduł odbierający oczekuje na dane od
klienta. Każdy wpis dziennika przesyłany jest przy użyciu pakietu
o~poniższych polach:

\begin{itemize}
\item wersja protokołu: aktualnie używana wersja protokołu komunikacyjnego,
\item kod CRC32: kod CRC32 bieżącego pakietu,
\item stempel czasu: stempel czasu pochodzący z~pakietu
  inicjalizującego przesłanego klientowi,
\item kod statusu: kod stanu usługi/hosta powiązany z~przesyłanym wpisem
\item nazwa hosta: nazwa klienta, który podlegał sprawdzeniu. Nie jest
  konieczne aby był to ten sam klient, który dostarcza dane,
\item opis usługi: nazwa usługi, która podlegała sprawdzeniu lub posty
  napis jeśli sprawdzenie dotyczy hosta,
\item wynik sprawdzenia: napis wygenerowany przez wtyczkę, która
  dokonywała sprawdzenie, zawierajacy dodatkowe dane natemat stanu
  hosta/usługi
\end{itemize}

Pakiety są zaszyfrowane z~użyciem algorytmu oraz klucza symetrycznego
pochodzącego z~pliku konfiguracyjnego. Po odebraniu spodziewanej
ilości danych, następuje próba odszyfrowania odebranych
danych. Sprawdzenie poprawności odebranych danych i~jednocześnie
weryfikacja uprawnień odbywa się poprzez kontrolę zawartości pola
CRC32. Jeśli wartość znajdująca sie w~tym polu, zgadza się z wartością
wyliczoną dla całości otrzymanych danych, to pakiet jest przyjmowany,
w~przeciwnym zaś razie pakiet zostanie odrzucony. Dalsze przetwarzanie
otrzymanego pakietu rozpoczyna się od porównania bierzącego stempla
czasu z~tym pochodzącym z odebranego pakietu. Jeśli różnica pomiędzy
nini jest zbyt duża, dane zostają odrzucane. Ostatnią czynnością
wykonywaną przez moduł odbierajacy jest zapisanie odebranego wpisu do
pliku zewnętrznych komend jądra systemu monitorującego.

Warto wspomnieć, że stempel czasu przesłany przez klienta nie jest
dostarczany do jądra monitorujacego. Służy on jedynie określeniu
odstępu czasu od inicjalizacji sesji do chwili otrzymania wiadomości~i
podjęciu decyzji o~otrzymaniu, bądź odrzuceniu pakietu. Do systemu
monitorującego trafia natomiast bierzący stempel czasu lokalnego
serwera, na którym uruchomiony jest moduł odbierajacy i~jądro systemu
monitorujacego. Istotną, może się również okazać informacja, iż
protokół komunikacyjny nie przewiduje przesyłania
ACK\footnote{ang. {\em Acknowledgement} -- pozytywne potwierdzenie,
  powszechnie przyjęta nazwa komunikatu potwierdzającego przyjęcie i
  przetworzenie danych przez aplikację}, bądź też
NACK\footnote{ang. {\em Negative-acknowledgement} -- potwierdzenie
  negatywne, powszechnie przyjęta nazwa komunikatu oznaczająca odmowę
  przyjęcia lub przetworzenia odebranych danych}. Oznacza to, iż moduł
wysyłający nie ma żadnej gwarancji ani informacji, że dane przesłane
do modułu odbierającego zostaną dostarczone do jądra systemu
monitorującego.

\section[Bezpieczeństwo][Bezpieczeństwo]{Bezpieczeństwo}

%bibliografia dodac libmcrypt
Bezpieczeństwo monitorowania z~użyciem dodatku NSCA opera się na
kryptografii symetrycznej oraz cyklicznym kodzie nadmiarowym
CRC32. Wiadomość inicjująca połączenie jest nieszyfrowana. Natomiast
każda wiadomość zawierająca wpisy dziennika jest zaszyfrowana
algorytmem wybranym podczas konfiguracji systemu. Dodatek NSCA
korzysta z~biblioteki libmcrypt i~umożliwia użycie jednego spośród
wielu algorytmów kryptografii symetrycznej, które zostały w niej
zaimplementowane. Użytkownik posiada jedynie możliwość wyboru
stosowanego algorytmu, natomiast jako tryb pracy stosowany jest tryb
sprzężenia zwrotnego szyfrogramu. Tryb ten wymaga zawsze inicjalizacji
zarówno kodera jak i~dekodera tym samym wektorem początkowym, którym
w~przypadku tego protokołu jest przesyłany przez serwer w pakiecie
inicjującym.

Wszystkie algorytmy symetryczne do prawidłowego działania wymagają,
aby komunikujące się strowny współdzieliły pewien sekret jakim jest
klucz używany do szyfrowania. Ujawnienie klucza symetrycznego wiąże
się z~kompromitacją całego systemu. W dodadktu NSCA klucz ten
uzyskiwany jest z~hasła, które musi być zapisane przez administratora
systemu zarówno w części odbierającej jak i~wysyłającej. Oczywistym
jest, iż poza współdzieleniem klucza, wszystkie komunikujące się węzły
muszą używać tego samego algorytmu kryptograficznego.

%zrodlo z kodem crc32
Algorytmy szyfrowania zapewniają tajność przesyłanej wiadomości,
jednak w~przypadku systemu monitorowania potrzebne jest również
zapewnienie integralności wiadomości. Integralność w~dodatku NSCA
zapewniana jest poprzez cykliczny kod nadmiarowy CRC32. Obliczanie
kodu CRC32 odbywa się poprzez dzielenie przesyłanego ciągu bitów przez
dzielnik o~długości 33 bitów, co daje kod CRC o~długości 32
bitów. W~celu sprawdzenia integralności, otrzymane bity są dzielone
przez kod CRC. Jeśli reszta z~dzielenia jest zero, oznacza to poprawna
weryfikację integralności wiadomości. Jeśli reszta z dzielenia jest
niezerowa oznacza to naruszenie integralności przesłanej wiadomości. W
szczególności, taka sytuacja może się zdarzyć, gdy klient używa innego
algorytmu kryptograficznego lub klucza. Pakiety, których integralność
nie zostanie pozytywanie zweryfikowana są odrzucane.

%zrodło o zagrozeniach z kodu CRC32
%stack overflow crc32 colision
Kryptografia zastosowana w~dodatku NSCA ma bardzo wiele wad. Najwięszą
z~nich jest zastosowanie kodu CRC32 do sprawdzania integralności
przesyłanych wiadomości. Kod ten można bardzo prosto i~szybko
obliczyć, a~ponadto posiada on niewielką długość. Niestety jest on
bardzo podatny na kolizje przez co nie powinien on być stosowany
w~kryptogrfii. Prawdopodobieństwo nie znalezienia kolizji po 200~000
prób wynosi poniżej~1\%. Oznacza to iż jedynie w niespełna 1\%
przypadków koniecne będzie obliczenie więcej niż 200~000 kodów CRC
przed znalezieniem kolizji. Prawdopodobieństwo nie znalezienia kolizji
w~zalezności od liczby obliczonych kodów CRC32 przedstawiono
w~\ref{tab:CRC32Colisions}. Łatwość odnalezienia kolizji nie jest
jedyną wadą modelu bezpieczeństwa zastosowanego w dodatku NSCA. Warto
przypomnieć, iż wszystkie ustawienia zarówno moduły wysyłającego jak
i~odbierającego przechowywane są w plikach na dyskach odpowiednich
urządzeń. Pliki te zawierają również klucze symetryczne, które są
stosowane w całym systemie. Oznacza to iż uzyskanie dostępu typu
odczyt do takiego pliku powoduje utratę tajności danych przesyłanych w
całym systemie. Ponadto przyjęty model bezpieczeństwa, nie zawiera
żadnej weryfikacji danych pochodzących od klientów. Oznacza to, że
każdy klient moze przesłać wpisy dziennka, udające wpisy pochodzące od
zupełnie innych klientów. W~szczególności jeśli atakujący uzyska klucz
symetryczny, to nie tylko będzie mógł odczytywać informacje o wpisach
przesyłanych od klientów, lecz także podszywać się pod klientów
i~przesyłać fałszywe wpisy. Taka luka może być wykorzystana przy ataku
na jakąś usługę. Atakujący rozpoczyna atak, po czym przechwytuje
pakiety z~wpisami z~dziennika, które mogą świadczyć o~rozpoczęciu
ataku i~w~zamian przesyła do serwera fałszywe pakiety informujące iż
wszystkie usługi pracują normalnie.

\begin{table}
\centering
\caption{Prawdopodopieństwo nie znalezienia kolizji w zależności od
  liczby obliczonych kodów CRC32}
\label{tab:CRC32Colisions}
\begin{tabular}{|c|c|}
\hline
Liczba obliczeń & Prawdopodobieństwo \\
\hline
50~000 & 74,7\% \\
\hline
77~000 & 50,1\% \\
\hline
78~000 & 49,2\% \\
\hline
102~000 & 29,8\% \\
\hline
110~000 & 24,5\% \\
\hline
128~000 & 14,8\% \\
\hline
150~000 & 7,3\% \\
\hline
200~000 & 0,95\% \\
\hline
\end{tabular}
\end{table} 

\section[Problemy][Problemy z monitorowaniem klienta mobilnego]{Problemy z monitorowaniem klienta mobilnego}

Dodatek NSCA jest powszechnie do monitorowania serwerów znajdujących
się za zaporą, która uniemożliwia wykonywanie aktywnych sprawdzeń lub
gdy charakterystyka monitorowanego parametru nie jest przystająca do
cyklicznego odpytywania. Dodatek ten może być stosowany, w~sieciach
o~statycznym charakterze, gdzie połączenia są stałe, a~łączność nie
ulega częstym przerwanią. Ponadto należy być świadomym słabości
kryptografii stosowanej w protokole wymiany danych. Stosowanie dodatku
NSCA poza zamkniętymi sieciami firmowymi może okazać się nieskuteczne
i~zawodne.

Problem monitorowania klienta mobilnego został szczegółowo opisany
w~\ref{chap:Wymagania}. Niestety dodatek NSCA nie spełnia bardzo wielu
z~przedstawionych wymagań przez co nie powinien być on stosowany
w~systemach tego typu. Głównymi problemami, który dyskryminują dodatek
NSCA w zastosowaniach do monitorowania klienta mbilnego są:

\begin{itemize}
\item Bezpieczeństwo: mechanizmy bezpieczeństwa zawarte w~protokole
  wymiany dancyh posiadają bardzo powazne luki. Zastosowanie CRC32 do
  sprawdzania spójności dancyh niesie za sobą bardzo duże
  ryzyko. Ponadto konieczność przechowywania na urządzeniu klucza
  symetrycznego, który kompromituje cały system znacząco osłaba
  stosowane mechanizmy bezpieczeństwa.
\item Nadpisywanie stempla czasu: Moduł odbierający dodaje do każdego
  wpisu dziennika aktualny stempel czasy. Powoduje to brak możliwości
  przesyłania, historycznych danych zgromadzonych w skutek utraty
  dostępu do sieci.
\item Brak dodatkowych mechanizmów uwierzytelnienia klienta: decyzja
  o~przydzieleniu klientowi dostępu czyli akceptacji przesłanych przez
  niego wpisów dziennika podejmowana jest na podstawie znajomości
  przez niego algorytmu szyfrowania oraz klucza.
\item Brak kontroli otrzymywanych danych: każdy klient, który zna
  klucz może przesyłać wpisy dotyczące dowolnego hosta i~dowolnej
  usługi. Brak jest mechanizmu, który pozwolił by na kontrolę tego,
  jaki klient ma prawo informować o~jakim hoście czy też usłudze.
\item Brak potwierdzenia dostarczenia danych: klient wysyłający dane
  nie ma żadnej informacji o~tym, czy jego dane zostały zaakceptowane
  czy odrzucone. Oznacza to brak możliwości synchronizacji danych na
  kliencie mobilnym i serwerze, gdyż nigdy nie mamy gwarancji iż
  wysłane przez klienta dane zostały przetworzone przez dodatek NSCA.
\item Brak implementacji dla systemów mobilnych: moduł wysyłający jest
  aktualnie zaimplementowany jedynie na systemy Windows oraz
  Linux. Wiele współczesnych urządzeń mobilnych, które powinny być
  monitorowane funkcjonuje z systemem Android czy też Windows Phone.
\item Przekazywanie danych tylko w~jedno miejsce: dane odebrane przez
  moduł odbierający mogą być przekazane jedynie w~jedno miejsce. Przy
  bardziej złożonych systemach, konieczna jest możliwość przekazywania
  dancyh do kilku systemów oraz definiowania reguł, które dane gdzie
  powinny trafić.
\end{itemize}

Powyższe wady zdecydowanie dyskryminują dodatek NSCA jako narzędzie do
monitoringu klienta mobilnego. W~związku z~powyższym w tej pracy
zaproponowano nowy protokół komunikacyjny, który został opisany
w~\ref{chap:ProtKom} oraz cały rzykładowy system do monitorowania
zarówno klientów stacjonarnych jak i~mobilnych opisany w~\ref{chap:ArchCal}.
