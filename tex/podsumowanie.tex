\chapter{Podsumowanie}
\label{chap:Podsumowanie}

W~ramach tej pracy wykonano projekt oraz implementację rozszerzenia
systemu Icinga na potrzeby monitorowania klientów mobilnych.
Zaprojektowany system został oparty na istniejących już elementach,
których wykorzystanie pozwoliło na uzyskanie bardzo rozbudowanej
funkcjonalności stosunkowo ograniczając nakład niezbędnej pracy.

Przed wykonaniem projektu przeprowadzono analizę dostępnych na rynku
systemów monitorujących. Zidentyfikowane problemy monitorowania
klientów statycznych oraz szczególny charakter klientów mobilnych
wskazały na potrzebę opracowania nowych rozwiązań.  W~ramach analizy
przedstawiono podstawowe możliwości otwartoźódłowych systemów {\em
  Cacti}, {\em Nagios} oraz {\em Icinga}. Porównanie tych systemów
wykazało, że najlepszym systemem do rozbudowy w~ramach tej pracy
będzie system {\em Icinga}.  Przedstawiono wymagania, jakie powinny
zostać spełnione przez projektowany system monitorujący, aby
umożliwiał on efektywne monitorowanie systemów mobilnych.

Kolejnym etapem przygotowania projektu była dokładna analiza
możliwości systemu {\em Icinga} w~kontekście zdefiniowanych
wymagań. W~ramach tej analizy przedstawiono ogólną architekturę
systemu oraz zestaw dopuszczalnych jego konfiguracji. Dokonano również
analizy dostępnych dodatków w~kontekście ich funkcjonalności i~jakości
ich wykonania. Ze względu na brak gotowych rozwiązań pozwalających na
monitorowanie klienta mobilnego zdefiniowano własny bezpieczny
protokół komunikacyjny. Zapewnia on bezpieczny transport danych
pomiędzy urządzeniem mobilnym a~serwerem monitorującym. Konieczne było
również wykonanie odpowiedniego dodatku do systemu {\em Icinga}, który
umożliwiłby odbiór danych od klientów mobilnych i~przekazanie ich do
miejsc zgodnych z~polityką całego systemu
monitorującego. 

Na podstawie projektu wykonano w~języku C++ implementację dodatku
NSCAv2. W~trakcie prac wykorzystany został szkielet aplikacji {\em
  Qt}, a~także biblioteka {\em boost}. Wszystkie algorytmy
kryptograficzne wymagane do implementacji protokółu komunikacyjnego
zostały zaimplementowane z~użyciem biblioteki {\em Crypto++}.

Końcowym etapem tej pracy było wdrożenie przykładowej konfiguracji
systemu {\em Icinga} zgodnie z~przedstawionym projektem i~z~użyciem
zaimplementowanego dodatku. Środowisko testowe zostało wykonane
w~domowej sieci lokalnej, jednak monitorowaniu podlegały również
publiczne serwery {\em google.com} oraz {\em
  mion.elka.pw.edu.pl}. Dzięki życzliwości Pana Marcina Kubika, który
wykonał implementację aplikacji mobilnej przeznaczonej na platformę
Android, możliwe było włączenie do środowiska testowego również dwóch
urządzeń pracujących pod kontrolą tego systemu.

Instancja systemu {\em Icinga} funkcjonowała nieprzerwanie przez trzy
tygodnie, co zapewniło zgromadzenie reprezentatywnego zbioru
danych. Wykorzystanie zewnętrznej infrastruktury zarówno do
monitorowania publicznych serwerów, jak i~przekazywania danych od
klientów mobilnych pozwoliło na symulację rzeczywistych warunków pracy
takiego systemu.

Zebrane dane potwierdziły przydatność zaprojektowanego systemu. Na ich
podstawie wskazano istotne trendy występujące w~sieciach
publicznych. Przeprowadzone testy wykazały również przydatność systemu
monitorującego klienta mobilnego. Dane zebrane w~trakcie monitorowania
takiego urządzenia mogą posłużyć dużym firmom do redukcji kosztów
utrzymania tych urządzeń i~optymalizacji posiadanej infrastruktury
sieciowej.

Jest bardzo wiele możliwych dróg rozwoju wykonanego projektu. Pierwszą
z~nich jest wykonanie narzędzia umożliwiającego w~łatwy i~intuicyjny
sposób zarządzanie konfiguracją systemu monitorującego i~jego
dodatków. Drugą, zdecydowanie ważniejszą i~dającą większe możliwości
rozwoju systemu, jest wykonanie systemu eksperckiego. System taki
mógłby na podstawie zgromadzonych danych historycznych (awariach,
parametrach wydajnościowych, zdarzeniach itp.) wykrywać z
wyprzedzeniem i~ostrzegać o~możliwości zaistnienia określonych
sytuacji. Pozwoliłoby to administratorowi podejmować odpowiednie kroki
zaradcze.
