\chapter{Podsumowanie}
\label{chap:Podsumowanie}

W~ramach tej pracy wykonano projekt oraz implementację systemu
monitorowania klientów statycznych i~mobilnych. Zaprojektowany system
został oparty na istniejących już elementach, których wykorzystanie
pozwoliło na uzyskanie bardzo rozbudowanej funkcjonalności stosunkowo
niskim nakładem pracy.

Przed wykonaniem projektu przeprowadzono analizę dostępnych na rynku
systemów monitorujących. W~ramach tej analizy przedstawiono podstawowe
możliwości systemu Cacti, Nagios oraz {\em Icinga}. Przeprowadzone
porównanie systemów wykazało, że najlepszym systemem do rozbudowy
w~ramach tej pracy będzie system {\em Icinga}.

Przed wykonaniem szczegółowego projektu systemu konieczne było
określenie docelowego zastosowania systemu. Zdefiniowano w~sposób
jasny i~czytelny problematykę monitorowania zarówno klientów
mobilnych, jak i~statycznych. Na podstawie przedstawionych
charakterystyk obu typów urządzeń sporządzono wymagania, jakie powinny
zostać spełnione przez projektowany system.

Kolejnym etapem przygotowania projektu była dokładna analiza
możliwości systemu {\em Icinga} w~kontekście zdefiniowanych
wymagań. W~ramach tej analizy przedstawiono ogólną architekturę
systemu oraz zestaw dopuszczalnych jego konfiguracji. Dokonano również
analizy dostępnych dodatków w~kontekście ich funkcjonalności i~jakości
ich wykonania. Analiza ta wykazała, iż system {\em Icinga} nie posiada
żadnych mechanizmów wspierających monitorowanie klienta mobilnego,
a~narzędzia przeznaczone dla klientów statycznych nie spełniałją
specyficznych wymagań dla projektowanego systemu.

Po dokładnym rozpoznaniu dostępnych elementów do budowy kompletnego
rozwiązania wykonano projekt całościowej architektury
systemu. Przedstawiono wybraną konfigurację systemu {\em Icinga}, a także
zbiór dodatków koniecznych do spełnienia wymagań. Ze względu na brak
gotowych rozwiązań pozwalających na monitorowanie klienta mobilnego
zdefiniowano własny bezpieczny protokół komunikacyjny. Zapewni on
transport danych pomiędzy urządzeniem mobilnym, a~serwerem
monitorującym. Konieczne było również wykonanie odpowiedniego dodatku
do systemu {\em Icinga}, który umożliwiłby odbiór danych od klientów
mobilnych i~przekazanie ich do miejsc zgodnych z~polityką całego
systemu. W szczególności tym miejscem docelowym jest system
monitorujący {\em Icinga}.

Na podstawie projektu wykonano implementację omawianego wcześniej
dodatku NSCAv2 w~języku C++. W~trakcie prac wykorzystany został
szkielet aplikacji Qt, a także biblioteka boost. Wszystkie algorytmy
kryptograficzne wymagane do implementacji protokółu komunikacyjnego
zostały zaimplementowane z~użyciem biblioteki Crypto++.

Końcowym etapem tej pracy było wykonanie przykładowej konfiguracji
systemu zgodnie z~przedstawionym projektem i~z~użyciem
zaimplementowanego dodatku. Środowisko testowe zostało wykonane
w~domowej sieci lokalnej, jednak monitorowaniu podlegały również
publiczne serwery google.com oraz mion.elka.pw.edu.pl. Dzięki
życzliwości Pana Marcina Kubika, który wykonał implementację aplikacji
mobilnej przeznaczonego na platformę Android, możliwe było włączenie do
środowiska testowego również dwóch urządzeń pracujących pod kontrolą
tego systemu.

Środowisko testowe funkcjonowało nieprzerwanie przez trzy tygodnie, co
zapewniło zgromadzenie reprezentatywnego zbioru danych. Wykorzystanie
zewnętrznej infrastruktury zarówno do monitorowania publicznych
serwerów, jak i~przekazywania danych od klientów mobilnych pozwoliło na
symulację rzeczywistych warunków pracy takiego systemu.

Zebrane dane potwierdziły przydatność zaprojektowanego systemu. Na ich
podstawie wskazano istotne trendy występujące w~sieciach
publicznych. Przeprowadzone testy wykazały również przydatność systemu
monitorującego klienta mobilnego. Dane zebrane w~trakcie monitorowania
takiego urządzenia mogą posłużyć dużym firmom do redukcji kosztów
utrzymania tych urządzeń i optymalizacji posiadanej infrastruktury
sieciowej.

Jest bardzo wiele możliwych dróg rozwoju wykonanego projektu. Pierwszą
z~nich jest wykonanie narzędzia umożliwiającego w~łatwy i~intuicyjny
sposób zarządzanie konfiguracją systemu monitorującego i~jego
dodatków. Drugą, zdecydowanie ważniejszą i~dającą większe możliwości
rozwoju systemu, jest wykonanie systemu eksperckiego. System taki
mógłby na podstawie zgromadzonych danych o~historycznych awariach
ostrzegać o~możliwości zaistnienia awarii. Pozwoliłoby to
administratorowi podejmować odpowiednie kroki w~celu przygotowania
sieci na awarię. Dzięki takiemu systemowi możliwe byłoby wcześniejsze
wykonywanie zamówień na urządzenia sieciowe, jeszcze zanim przestaną
one działać. Ponadto warto rozważyć możliwość implementacji dodatku
pozwalającego na bardziej elastyczne gromadzenie informacji w~bazie
danych i~generację na ich podstawie wykresów.
