\chapter{Dostępne systemy monitorujące}
\label{chap:Systemy}

\section[Przegląd systemów][Przegląd systemów dostępnych na rynku]{Przegląd systemów dostępnych na rynku}

Na rynku dostępnych jest wiele bardzo różnych systemów
monitorujacych. Narzędzia z~tej grupy możemy podzielić na dwie
ketegorie:

\begin{itemize}
\item Systemy dostępnościowe
\item Systemu analityczne
\end{itemize}

Systemy monitorujące w~których główny nacisk jest położony na
zapewnienie ciągłej dostępności monitorowanych usług. Systemy te
wspierają administratora w~codziennych zadaniach, poprzez nieustanne
monitorowanie aktualnego stanu sieci. Narzędzia te sa wykorzystywane
przedewszystkim do szybkiego powiadamiania oraz lokalizacji awawarii.

Systemy analityczne, w~kontekście monitorowania infrastruktury
sieciowej, to systemy, które są nastawione na zbieranie i~analizę
posiadanych danych. Tego typu systemy nie są zazwyczaj wykorzystywane
do powiadamiania czy lokalizacji awarii, ich zadaniem jest
przedewszystkim gromadzenie danych dotyczących zużycia poszczególnych
zasobów, czy też wskaźników jakości poszczególnych usług. Systemy tego
typu posadają zazwyczaj bardzo rozbudowane narzędzia służące do
generacji i~analizy wykresów na podstawie zebranych wcześniej danych.

W ostatnich latach można zauważyć wzrost popularności rozwiązań
hybrydowych. Pozwalają one na kompleksowe zarządzanie infrastrukturą
sieciową. Dzięki zastosowaniu takiego systemu administrator uzyskuje
jeden interfejs, w którym może zarówno śledzić bierzący stan sieci
i~diagnozować awarie, jak również prowadzić analizę danych
historycznych.

Przechowywanie danych zgromadzonych podczas monitorowania może odbywać
się na różne sposoby. Podstawową techniką przechowywania danych,
jeszcze 5 lat temu były płaskie pliki zawirające zgromadzone
dane. Rozwiązanie tego typu jest bardzo uciążliwe, a~sprawne
zarządzanie zgromadzonymi danymi wymaga dużego wkładu pracy własnej
administratora. Obecnie rozpowszchniają się techniki przechowywania
zebranych danych w~oparciu o~bazy danych. Współcześnie używane typy baz danych to:

\begin{itemize}
\item Relacyjne bazy danych
\item Cykliczne bazy danych\footnote{ang. {\em Round Robin Database}}
\end{itemize}

Dane przechowywane w~relacyjnych bazach danych zorganizowane są
w~postaci tabel, a~powiązania pomiędzy danymi nazywane są
relacjami. Taka organizacja bazy danych sprawia, że baza danych w
której gromadzone są wyniki wraz z~upływem czasu rośnie. Powoduje to
zwiększenie zajętości przestrzeni dyskowej, a~także wpływa na czas
wykonywania operacji. Dane są przechowywane w~bazie do czasu, gdy
użytkownik jawnie je usunie. Pozwala to na przeglądanie dowolnie
długiego okresu historii, bez utraty dokładności, a także na
dynamiczne zarządzanie czasem przechowywania danych.

Cykliczne bazy danych posiadają natomiast stały, definiowany podczas
tworzenia rozmiar. Rozmiar ten określa o liczbę porcji danych jaka
może być przechowywana w bazie. Jeśli rozmiar bazy przekroczy rozmiar
zadany przy tworzeniu, wykonywana jest konsolidacja danych. Polega ona
na wyliczeniu zadanych wartości w~odpowiednich przedziałach
i~zachowanie ich w~pojedyńczych rekordach, a~usunięcie dokładnych
danych. Możliwe są trzy typy konsolidacji danych, minimum, średnia
oraz maksimum. Rozmiar bazy danych jest definiowany w~chwili jej
tworzenia i~późniejsza modyfikacja tego rozmiaru nie jest już
możliwa. Ponadto nalezy zwrócić szczególną uwagę, na fakt iż dane są
usuwane z~bazy danych bez wiedzy użytkownika, przez co taka baza
danych nie może zostać użyta do dokładnej analizy danych
historycznych.

Każdy typ systemu, jak i~rodzaj bazy danych posiada swoje
zastosowanie. Należy zatem rozważnie zdefiniować wymagania jakie
stawia się przed systemem. Po dokonaniu ich analizy i~analizy
możliwości konkretnego systemu dokonać wyboru systemu, który system
najlepiej spełnia przedstaiwone wymagania.

\subsection[Cacti][System monitorowania Cacti]{System monitorowania Cacti}

%dodac do bibliografii http://oss.oetiker.ch/rrdtool/
Jest to system monitorujacy, rozwijany przez The Cacti Group
Inc. i~dystrybuowany na licencji~GPL\footnote{ ang. {\em General
    Public Licens} - popularna licencja oprogramowania o~otwartych
  źródłach. Treść licencji mozna znaleźć w XXX}. System bazuje na
narzędziu RRDtool. Jest to narzędzie, które pozwala na wykorzystanie
cyklicznej bazy danych do składowania pomiarów wartości w zadanym
przedziale czasowym. Ponadto narzędzie dostarcza funkcji do generacji
wykresów w~kilku formatach. Dzięki wykorzystaniu wspomnianego
narzędzia system ma bardzo prostą budowę i~składa się z~następujących
elementów:

\begin{itemize}
\item interfejs użytkownika,
\item dostawca danych.
\end{itemize}

%dodać zdjęcie interface
Interfejs użytkownika został napisany w~języku PHP. Do jego działania
niezbędny jest serwer http np. Apache. Z~poziomu interfejsu
użytkownika możliwa jest graficzna konfiguracja całego
systemu. Interfejs posiada klasyczna budowę. Składa się on
z~jednokolorowego paska menu, w~którym zawarte są odnośniki do
poszczególnych podstron oraz z~pulpitu, na którym wyświetlane są
wybrane dane. Interfejs umożliwia graficzne przedstawienie wyników
w~postaci wykresów. Format wykresu może być definiowany bezpośrednio
przez użytkownika, lub można skorzystać z~bogatej biblioteki gotowych
szablonów. Dostęp do interfejsu zabezpieczony jest poprzez mechanizm
uwierzytelnienia użytkownika systemu monitorującego. Mozliwe jest
definiowanie wielu użytkowników oraz ich uprawnienia. Każdy użytkownik
ma możliwość definiowania własnego zestawu wykresów oraz pulpitów.

%bibliografia snmp
Dostawca danych jest to element systemu, który jest odpowiedzialny za
faktyczne wykonywanie sprawdzeń danej wartości i~przekazywanie ich do
narzędzia RRDTool. System umożliwia wybór jednego z~dwóch dostawców
danych. Pierwszym z~nich jest cmd.php, który jest prostym skryptem
napisanym w języku php. Umożliwia on monitorowanie aktywne urządzeń
przy pomocy protokołu SNMP\footnote{ {\em Simple Network Management
    Protocol} -- protokół zarządzania urządzeniami sieciowymi
  i~uzyskiwania informacji o~ich stanie. Zorganizowany w~formie
  drzewa, gdzie każdy liść posiada globalnie unikalny identyfikator
  o~ściśle określonym znaczeniu. Szeroko opisany w XXX}. Skrypt
cmd.php przeznaczony jest do monitorowania jedynie niewielkich
sieci. Ze względów wydajnosciowych, nie jest możliwe wykorzystanie go
do monitorowania rozległej infrastruktury.

Drugim z~możliwych do wyboru dostawców danych jest program Spine,
nazywany również Cacttid. Jest to program napisany w~języku~C, który
uruchomiony jest jako serwis systemowy na urządzeniu
monitorujacym. Umożliwa on monitorowanie urządzeń zarówno poprzez
protokół SNMP jak i~z~wykorzystaniem innych metod. Możliwośc
dostarczenia własnych metod monitorowania opiera się na dostarczeniu
skryptu lub pliku wykonywalnego, który będzie cyklicznie uruchamiany
przez Cactid, a~jako wyniki przekazywane w taki sam sposób jak
z~sprawdzeń operających się na SNMP.

Żaden z~dostawców danych nie umożliwia monitorowania danego urządzenia
lub usługi w~sposób pasywny. Cacti nie posiada również żadnego
mechanizmu, który pozwoliłby na monitorowanie sieci w~sposób
rozproszony. Oznacza to, iż administrator musi zmienić konfigurację
sieci, tak aby jeden serwer miał dostęp do każdego urządzenia, lub
konfigurować i~zarządzać osobną instancją w~każdym serwerze. Jest to
bardzo niewygodne i~wręcz uniemożliwia monitorowania rozległych sieci
przy pomocy Cacti.

\subsection[Nagios][System monitorowania Nagios]{System monitorowania Nagios}

System Nagios został opublikowany w~1999 na licencji GPL. System od
niemal 15 lat jest ciągle rozwijany i~udoskonalany, zarówno przez
autorów jak i~przez szeroką społeczność. W~systemie Nagios najwyższym
priorytetem jest dbałość o~zapewnienie dostępności wszystkich
monitorowanych usług. Organizacja systemu zakłada, iż w~sieci znajdują
się urządzenia, które mogą świadczyć pewne usługi. Każde urządzenie
jak i~usługa może być w jednym z trzech stanów:

\begin{description}
\item[OK] usługa działa poprawnie
\item[WARNING] monitorowane parametry przekroczyły stan ostrzegawczy
\item[CRITICAL] parametry usługi przekroczyły stan krytyczny, usługa
  lub urządzenie nie funkcjonuje
\end{description}

System posiada rozbudowane algorytmy określania stanu każdego
urządzenia oraz usługi. Działanie usługi, jest zawsze zależne od stanu
urządzenie, na którym dana usługa jest świadczona. Ponadto użytkownik
może definiować zależności pomiędzy urządzeniami. System Nagios
posiada rozbudowany system powiadamiania administratora o~wystąpieniu
awarii oraz o~jej zakończeniu, lub innych zdefiniowanych wydarzeniach
systemowych. Ponadto mozliwe jest automatyczne wykonowanie
zdefiniowanych programów lub skryptów, jeśli wystąpiło jakieś
zdarzenie. Podstawowa wersja systemu składa się z następujących
elementów:

\begin{itemize}
\item Interfejs graficzny
\item Rdzeń monitorujący
\end{itemize}

%odnosnik do CGI
%foto interface
Interfejs graficzny został napisany w~języku~C z~wykorzystaniem
technologi CGI\footnote{{\em Common Gateway Interface} --
  znormalizowany interfejs służący do komunikacji pomiędzy serwerem
  www, a~zewnętrznymi programami. Inerfejs ten jest wykorzystywany do
  generowania stron internetowych na żadanie klienta. Zewnętrzny
  program generuje stronę w~języku html, a~następnie serwer przesyła
  ją do klienta. Szczegółowy opis można znaleźć w XXX}. Jego wygląd
jest zgodny z~standardami z~lat 90. Klasyczna strona WWW bez
dynamicznie zmieniającej się treści. Dane odświeżane są na żądanie
klienta, lub co określony czas. Wykorzystana technologia zakłada
przesyłanie za każdym razem całego dokumentu html do klienta,
w~związku z~czym generowany jest nadmierny ruch sieciowy. Widok
użytkownika składa sie z~kliku części. Po lewej stronie widoczne jest
klasyczne menu, umożliwiające użytkownikowi wybór treści. Na górze
strony natomiast znajduje się podsumowanie aktualnego stanu
monitorowanych urządzeń i~usług. Centralną częśc okna zajmuje pulpit,
który prezentuje uzytkownikowi treść wybraną wcześniej
z~menu. Interfejs użytkownika umożliwia podgląd aktualnego stanu usług
oraz urządzeń. Informacja ta może być wyświetlana w~formie listy
zawierającej urządzenie i~usługi, lub w~postaci mapy sieci, która
pozwala na monitorowanie stanu urządzenia w~korelacji z~jego logicznym
umieszczeniem w~strukturze sieciowej. Możliwe jest również
przeglądanie histori awarii oraz prostych wykresów zależności stanu
urządzenia lub usługi w~zadanym przedziale czasu. Dostęp do interfejsu
chroniony jest przy pomocy autoryzacji uwierzytelnienia http. Możliwe
jest definiowanie wielu użytkowników, jednak tylko z~poziomu
urządzenia na którym uruchomiony jest system monitorujacy. Należy
zauważyć również, że wszyscy użytkownicy posiadają takie same
uprawnienia do wyświetlania danych oraz zarządzania.

%referencja do tutoriala o pluginach
Rdzeń monitorujący został zaiplmenetowany w~języku C. Jest to centrum
całego systemu, gdyż zajmuje się on przetwarzaniem wszystkich
bieżących danych monitorowania, a~następnie składowaniem ich
w~plikach. Ta cześć systemu jest odpowiedzialna za wykonywanie
sprawdzeń w~określonych odstępach czasu. Każde sprawdzenie odbywa się
poprzez wykonanie komendy zdefiniowanej przez użytkownika. Komenda ta
może zawierać zarówno wykonanie pliku binarnego jak i~dowolnego
skryptu. W~ramach projektu Nagios, rozwijany jest zestaw
wtyczek\footnote{Należy zwrócić uwagę na różne znaczenie słów wtyczka
  (ang. {\em Plugin}) oraz dodatek (ang. {\em Addon}). }, czyli
programów służących do monitorowania podstawowych usług oraz
parametrów urządzeń. Dostępna jest bardzo duża liczba wtyczek, dzięki
czemu system Nagios moze monitorować w~sposób aktywny wszystkie
podstawowe parametry lub usługi. System umożliwa również monitorowanie
dowolnych usług w~sposób pasywny.

System posiada rozbudowane możliwosci monitorowania
rozproszonego. Niestety, do wykonania znacznej części z~tych
konfiguracji potrzebne są elementy systemu, które są dystrybuowane za
opłatą. Istnieją również darmowe dodatki, które pozwalają na
przechowywanie zgromadzonych danych zarówno w~bazie relacyjnej jak
i~cyklicznej. Możliwa jest również częściowa integracja systemu Nagios
z~dodatkami lub systemami, które pozwalają na wizualizacje
zgromadzonych danych.

\subsection[Icinga][System monitorowania Icinga]{System monitorowania Icinga}
\label{subsec:Icinga}

System Icinga powstał w 2009 roku jako klon (ang. fork) systemu
Nagios. System został wzbogacony o~wiele nowych elementów, a~także
poprawiono wiele błędów obecnych w~systemie Nagios. Dzięki zachowaniu
wstecznej kompatybilności zarówno wszystkie wtyczki jak i~dodatki
systemu Nagios mogą być wykorzystane w~systemie Icinga. Pozyskano temu
bardzo dużą bazę wtyczek, co umożliwia monitorowanie tych samych usług
i~urządzeń co przodek.

%refererencja do agavi http://www.agavi.org/ no i do Ajaxa
System Icinga został wyposażony w zupełnie nowy interfejs
graficzny. Został on zaimplementowany w jezyku PHP przy użyciu
szkieletu aplikacji agavi. Jest on zatem oparty na technologii Ajax,
dzięki której komunikacja z~użytkownikiem, nie opera się na
przesyłaniu całych stron w jezyku html, lecz na realizacji żądań
generowanych poprzez język skryptowy wykonywany po stronie
użytkownika. Dzięki zastosowaniu tej technologi, proces wyświetlania
strony zużywa mniejszą część pasma, a~serwer został odciążony. Nowy
interfejs użytkownika jest w~pełni dynamiczny, składa się on
z~rozszrzalnego menu po lewej stronie oraz pulpitów użytkownika
w~centralnej cześci. Możliwe jest otwieranie wielu pulpitów oraz
wyświetlanie poszczególnych informacji w~osobnych oknach, które można
swobodnie przemieszczać w~obszarze strony. Znacznej zmianie uległ
również model bezpieczeństwa. W~nowym interfejsie graficznym, każdy
użytkownik, posiada swój zestaw zdefiniowanych uprawnień. Oznacza to,
że możliwe jest ograniczenie użytkownikowi dostępu do danych
o~oknkretnej usłudzie lub zabronić wykonywania niektórych czynności
administracyjnych. Zarządzanie użytkownikami oraz ich uprawnieniami
możliwe jest również z~poziomu graficznego interfejsu użytkownika, co
znacząco podosi wygodę użytkowania systemu.

Kolejną istotną różnicą, jest zmiana architektury systemu. System
Nagios posiada budowę monolityczną, a~współpraca pomiedzy
poszczególnymi jego komponentami odbywa się w~sposób bardzo zawiły
i~niejednorodny. System Icinga wprowadził natomiast budowę
modularną. Wszystkie możliwe komponenty systemu zostały wyodrębnione,
a~do swobodnej komunikacji pomędzi nimi zdefiniowano wygodne API. Taka
budowa umożliwia przedewszystki rozmieszczenie poszczególnych
komponentów systemu na różnych fizycznych maszynach, co w~przypadku
duzych sieci moze spowodować znaczący wzrost wydajnośći
i~niezawodnosci. Dostarczenie jednolitego REST API\footnote{dodac
  notke o tym} umożliwia również prostsze tworzenie dodatków
rozbudowujacych możliwosci systemu.  W~systemie Icinga rozbudowano
także możliwosci współpracy z~bazą danych. System ten umożliwa
współpracę, już nie tylko z~bazą MySQL, lecz również z~bazami
PostgreSQL czy też z~systemem zarządzania bazą danych firmy
Oracle. Możliwość wykorzystania bazy danych Oracle, jest bardzo
istotna, jeśli dane dotyczące pomiarów muszą być przechowywane przez
długi czas, lub jeśli monitorowana infrastruktura jest bardzo
rozbudowana.

System Icinga, nie tylko umożliwia rozmieszczenie modułów na różnych
fizycznych maszynach, lecz również umożliwia wiele innych
konfiguracji, które można wykorzystać podczas monitorowania
rozproszonego. Szczególnie wartą zauważenia jest konfiguracja,
w~której występuje wiele równorzędnych instancji rdzenia
monitorującego, natomiast wszystkie współpracują używając jednej bazy
danych. Centralna baza danych stanowi źródło danych dla interfejsu
graficznego. Taka konfiguracja umożliwa monitorowanie bardzo rozległej
lub wielosegmentowej infrastruktury. Należy również nadmienić, iż
wszystkie elementy niezbędne do konfiguracji takiego rozwiązania są
darmowe.

\section[Podsumowanie][Podsumowanie]{Podsumowanie}

Współczesne systemy monitoringu, są bardzo bogato wyposażone
i~posiadają szereg zaawansowanych możliwosci. Każdy z~systemów oferuje
unikalny zestaw rozwiązań, które z~pewnością mogą zostać wykorzystane
w~wielu instytucjach. Porównując wszystkie omówione systemy, należy
zwrócić szczególną uwagę, na różnice w~ich możliwych zastosowaniach
docelowych.

Systemy, takie jak Cacti zaliczane są do grupy systemów
analiztycznych. Ich celem jest zatem zapewnienie możliwości
gromadzenia oraz analizy danych. Zbierane dane mają charakter
pojedynczych, dokładnych wartości, na podstawie których prezentowane
są użytkownikowi odpowiednie wykresy. Niestety ze względu na sposób
gromadzenia danych - protokół SNMP, oraz ubogość metod ich gromadzenia
systemy te, nie mogą być wzbogacone o~funkcjonalność charakterystyczną
dla systemów dostępnościowych.

Drugą grupę systemów stanowią natomiast systemy dostępnościowe, takie
jak Nagios czy Icinga. Ich głównym celem jest monitorowanie bieżącego
stanu infrastruktury i~raportowanie użytkownikowi najświeższych
informacji. Systemy te zostały również zaprojektowane, aby wspomagać
administratora w~lokalizacji awarii. Głównym typem danych na których
operują te systemy jest stan urządzenia lub usługi. Zdefiniowanie
odpowiednich poziomów kwantyzacji dla stanów pozwala na szybkie
uzyskiwanie poglądowych informacji o~stanie sieci. Podczas
monitorowania gromadzone są również dane szczegółowe. Ich
przetwarzaniem nie zajmują się już jednak same systemy monitorowania,
lecz liczne dodatki do nich. Możliwe jest zatem rozbudowanie systemu
tego typu, o~dodatkowe elementy, które pozwolą uzyskać system
hybrydowy. System taki bedzie mógł pełnić rolę zarówno systemu
dostepnościowego jak i analitycznego.

Wybierając system monitorujący, nalezy zatem dokonać szczegółowej
analizy wymagań stawianych przed systemem. Szczegółowe porównanie
wszystkich przedstawionych systemów monitorowania zawarto
w~\ref{tab:PorownanieSys}.

\begin{longtable}[c]{|p{4.5cm}||p{3cm}|p{3cm}|p{3cm}|}
  \caption[Porównanie systemów monitorowania]{Porównanie systemów monitorowania} \label{tab:PorownanieSys} \\
  \hline \multicolumn{1}{|c||}{Nazwa systemu} &
  \multicolumn{1}{c|}{Cacti} & \multicolumn{1}{c|}{Nagios} &
  \multicolumn{1}{c|}{Icinga} \tabularnewline \hline \hline
  \endfirsthead

  \multicolumn{4}{c}%
  {{\tablename\ \thetable{} -- Kontynuacja z~poprzedniej strony}} \\
  \hline
  \multicolumn{1}{|c||}{Nazwa systemu} &
  \multicolumn{1}{c|}{Cacti} & \multicolumn{1}{c|}{Nagios} &
  \multicolumn{1}{c|}{Icinga} \tabularnewline 
  \hline \hline
  \endhead

  \hline \multicolumn{4}{|r|}{{Kontynuacja na następnej stronie}} \\ \hline
  \endfoot

  \hline\hline
  \endlastfoot

  \raggedright{Podgląd stanu bieżącego} & \raggedright{Nie} &
  \raggedright{Ta}k & \raggedright{Ta}k \tabularnewline 
  \hline

  \raggedright{Podglad danych historycznych} &\raggedright{Tak} &
  \raggedright{Tak, przez dodatek} & \raggedright{Tak, przez dodatek}
  \tabularnewline
  \hline

  \raggedright{Dane w~formie wykresu} & \raggedright{Tak} &
  \raggedright{Tak, przez dodatek} & \raggedright{Tak, przez dodatek}
  \tabularnewline 
  \hline

  \raggedright{Przechowywanie danych w~bazie cyklicznej} & \raggedright{Tak} &
  \raggedright{Tak, przez dodatek} & \raggedright{Tak, przez dodatek}
  \tabularnewline
  \hline

  \raggedright{Przechowywanie danych w~bazie relacyjnej} & \raggedright{Nie} &
  \raggedright{Tak, przez dodatek} & \raggedright{Tak, przez dodatek}
  \tabularnewline
  \hline

  \raggedright{Powiadomienia o~awarii} & \raggedright{Nie} &
  \raggedright{Tak, email lub telefon} & \raggedright{Tak, email lub telefon}
  \tabularnewline
  \hline

  \raggedright{Wsparcie w~lokalizacji awarii} & \raggedright{Nie} &
  \raggedright{Tak, poprzez mapę logiczną sieci} & \raggedright{Tak, poprzez mapę logiczną sieci}
  \tabularnewline
  \hline

  \raggedright{Obsługa SNMP} & \raggedright{Tak} &
  \raggedright{Tak, przez wtyczkę} & \raggedright{Tak, przez wtyczkę}
  \tabularnewline
  \hline

  \raggedright{Zbieranie danych spoza SNMP} & \raggedright{Tak, niewielka liczba dostępnych metod} &
  \raggedright{Tak, bogaty zestaw wtyczek} & \raggedright{Tak, bogaty zestaw wtyczek}
  \tabularnewline
  \hline

  \raggedright{Monitorowanie pasywne} & \raggedright{Nie} &
  \raggedright{Tak} & \raggedright{Tak}
  \tabularnewline
  \hline

  \raggedright{Nowoczesny interfejs użytkownika} & \raggedright{Nie} &
  \raggedright{Nie} & \raggedright{Tak, z wykorzystaniem technologi AJAX}
  \tabularnewline
  \hline

  \raggedright{Wielu użytkowników} & \raggedright{Tak} &
  \raggedright{Tak} & \raggedright{Tak}
  \tabularnewline
  \hline

  \raggedright{Metoda uwierzytelnienia} & \raggedright{Uwierzytelnienie wewnętrzne} &
  \raggedright{Uwierzytelnienie http} & \raggedright{Uwierzytelnienie wewnętrzne}
  \tabularnewline
  \hline

  \raggedright{Zarządzanie kontami użytkowników z~interfejsu} & \raggedright{Tak} &
  \raggedright{Nie} & \raggedright{Tak}
  \tabularnewline
  \hline

  \raggedright{Definiowanie uprawnień dla uzytkowników} & \raggedright{Tak, przez interfejs graficzny} &
  \raggedright{Nie} & \raggedright{Tak, przez interfejs graficzny}
  \tabularnewline
  \hline

  \raggedright{Modularność} & \raggedright{Nie} &
  \raggedright{Nie} & \raggedright{Tak}
  \tabularnewline
  \hline

  \raggedright{Rozmieszczenie modułów na różnych urządzeniach fizycznych} & \raggedright{Nie dotyczy} &
  \raggedright{Nie dotyczy} & \raggedright{Tak}
  \tabularnewline
  \hline

  \raggedright{Możliwość monitorowania rozproszonego z~instancją
    nadrzędną} & \raggedright{Nie} & \raggedright{Tak} &
  \raggedright{Tak}
  \tabularnewline
  \hline

  \raggedright{Możliwość monitorowania rozproszonego bez instancji nadrzędnej} & \raggedright{Nie} &
  \raggedright{Tak, konieczny płatny dodatek} & \raggedright{Tak}
  \tabularnewline
  \hline

  \raggedright{Generacja raportów} & \raggedright{Nie} &
  \raggedright{Nie} & \raggedright{Tak, z~wykorzystaniem JasperReports}
  \tabularnewline
  \hline

  \raggedright{Możliwość monitorowania klienta mobilnego} & \raggedright{Nie} &
  \raggedright{Nie} & \raggedright{Nie}
  \tabularnewline
  \hline

  \raggedright{Dostępność} & \raggedright{Darmowy} &
  \raggedright{Częściowo darmowy, wiele płatnych elementów i~funkcjonalnosci} & \raggedright{Darmowy}
  \tabularnewline
  \hline

  \raggedright{Licencja} & \raggedright{GPL v2} &
  \raggedright{GPL v3 (tylko darmowe elementy)} & \raggedright{GPL v2}
  \tabularnewline
  \hline

\end{longtable}

Przedstawione systemy monitorujące w~znacznym stopniu zaspokajają
zaporzebowanie rynku na systemy monitorowania. Pojawia się jednak
pewna nisza związana z~monitorowaniem urządzeń mobilnych. Zadanie to
nie jest trywialne i~wymaga obecności dodatkowych mechanizmów zarówno
na urządzeniu mobilnym, jak i~w~innych elementach systemu. Żaden
z~analizowanych systemów nie posiadał w~swej implementacji ani
w~oficjalnych repozytoriach z~dodatkami, oprogramowania, które
pozwalałoby na monitorowanie parametrów urządzenia mobilnego.

