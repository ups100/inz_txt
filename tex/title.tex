
\begin{titlepage}
    % Strona tytułowa
    \vbox to\textheight{\hyphenpenalty=10000
    \begin{center}
	\begin{tabular}{p{107mm} p{9cm}}
	    \begin{minipage}{9cm}
	      \begin{center}
	      Politechnika Warszawska \\
	      Wydział Elektroniki i~Technik Informacyjnych \\
	      Instytut Informatyki
	      \end{center}
	    \end{minipage}
	    &
	    \begin{minipage}{8cm}
	    \begin{flushleft}
	     \footnotesize
	      Rok akademicki 2013/2014
	    \vspace*{2.75\baselineskip}
	    \end{flushleft}
	    \end{minipage} \\
	\end{tabular}
        \vspace*{2.5\baselineskip}
	\par\vspace{\smallskipamount}
	\begin{center}
		\includegraphics[angle=0,scale=1]{img/logo_pw.png}
	\end{center}
	\vspace*{2\baselineskip}{\LARGE Praca dyplomowa inżynierska\par}
	\vspace{3\baselineskip}{\LARGE\strut Krzysztof Opasiak\par}
	\vspace*{2\baselineskip}{\huge\bfseries Rozproszony monitoring systemów komputerowych\par}

	\vspace*{7\baselineskip}
	\hfill\mbox{}\par\vspace*{\baselineskip}\noindent
	\begin{tabular}[b]{@{}p{3cm}@{\ }l@{}}
	    {\large\hfill } & {\large }
	\end{tabular}
	\hfill
	\begin{tabular}[b]{@{}l@{}}
	Opiekun pracy: \\[\smallskipamount]
	{\large dr inż. Piotr Gawkowski}
	\end{tabular}\par
	\vspace*{4\baselineskip}
    \begin{tabular}{p{\textwidth}}
    \begin{flushleft}
	\begin{minipage}{7cm}
	Ocena \dotfill
	\par\vspace{1.6\baselineskip}
	\dotfill
	\par\noindent
	\centerline{\footnotesize Podpis Przewodniczącego} \par
	\centerline{\footnotesize Komisji Egzaminu Dyplomowego}\par
	\end{minipage}
    \end{flushleft}
    \end{tabular}
    \end{center}}

    % Życiorys
    \newpage\thispagestyle{empty}
    \begin{tabular}{p{4cm} p{12cm}}
    \begin{minipage}{5cm}
    
    \includegraphics[height=5cm,width=4cm]{img/foto.jpg}
    \end{minipage}
    &
    \begin{minipage}{12cm}
    \begin{flushleft}
    \par\noindent\vspace{1\baselineskip}
    \begin{tabular}[h]{l l}
    {\normalsize\it Kierunek:} & {\normalsize \hspace{0.8cm} Informatyka}
    \end{tabular}\par\noindent\vspace{1\baselineskip}
    \begin{tabular}[h]{l l}
    {\normalsize\it Specjalność:}& {\normalsize \hspace{0.4cm} Inżynieria Systemów Informatycznych}
    \end{tabular}
    \par\noindent\vspace{1\baselineskip}
    \begin{tabular}[h]{l l}
    {\normalsize\it Data urodzenia:} & {\normalsize \hspace{4.6cm} 1990.12.28}
    \end{tabular}
    \par\noindent\vspace{1\baselineskip}
    \begin{tabular}[h]{l l}
    {\normalsize\it Data rozpoczęcia studiów:} & {\normalsize \hspace{2.8cm} 2010.10.01}
    \end{tabular}
    \par\noindent\vspace{1\baselineskip}
    \end{flushleft}
    \end{minipage}
    \end{tabular}
    \vspace*{1\baselineskip}
    \begin{center}
	{\large\bfseries Życiorys}\par\bigskip
    \end{center}

    \indent
    Urodziłem się 28 grudnia 1990 w Koninie. Uczęszczałem do Szkoły Podstawowej numer 8
    im. Powstańców Wielkopolskich w Koninie. Następnie uczęszczałem do Gimnazjum Towarzystwa
    Salezjańskiego w Konienie.

    \par
    \indent
    W latach 2006-2010 uczęszczałem do Technikum w Zespole Szkół im. Mikołaja Kopernika w Koninie.
    W trakcie nauki w tej szkole dwukrotnie przyznano mi stypendium Prezesa Rady Ministrów za
    bardzo dobre wyniki w nauce oraz wzorowe zachowanie.
    W roku 2010 ukończyłem z wyróżnieniem szkołę średnią, a następnie zdałem maturę oraz egzamin
    zawodowy uzyskując tytuł Technik Teleinformatyk.

    \par
    \indent
    W październiku 2010 roku rozpocząłem studia stacjonarne pierwszego stopnia na Wydziale Elektroniki
    i Technik Informacyjnych na kirunku Informatyka.

    \vspace{2\baselineskip}
    \hfill\parbox{15em}{{\small\dotfill}\\[-.3ex]
    \centerline{\footnotesize podpis studenta}}\par
    \vspace{2\baselineskip}
    \begin{center}
 	{\large\bfseries Egzamin dyplomowy} \par\bigskip\bigskip
    \end{center}
    \par\noindent\vspace{1.5\baselineskip}
    Złożył egzamin dyplomowy w dn. \dotfill 20\_\_r
    \par\noindent\vspace{1.5\baselineskip}
    Z wynikiem \dotfill
    \par\noindent\vspace{1.5\baselineskip}
    Ogólny wynik studiów \dotfill
    \par\noindent\vspace{1.5\baselineskip}
    Dodatkowe wnioski i uwagi Komisji \dotfill
    \par\noindent\vspace{1.5\baselineskip}
    \dotfill

    % Streszczenie
    \newpage\thispagestyle{empty}
    \vspace*{2\baselineskip}
    \begin{center}
	{\large\bfseries Streszczenie}\par\bigskip
    \end{center}

    {\itshape
    Praca ta prezentuje \ldots}
    \vspace*{1\baselineskip}

    \noindent{\bf Słowa kluczowe}: {\itshape słowa kluczowe.}
    \par
    \vspace{4\baselineskip}
    \begin{center}
	{\large\bfseries Abstract}\par\bigskip
    \end{center}
    \noindent{\bf Title}: {\itshape Thesis title.}\par
    \vspace*{1\baselineskip}
    {\itshape
    This thesis describes \ldots}
    \vspace*{1\baselineskip}

    \noindent{\bf Key words}: {\itshape key words.}

\end{titlepage}

% ex: set tabstop=4 shiftwidth=4 softtabstop=4 noexpandtab fileformat=unix filetype=tex spelllang=pl,en spell:
