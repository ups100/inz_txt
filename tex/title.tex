
\begin{titlepage}
    % Strona tytułowa
    \vbox to\textheight{\hyphenpenalty=10000
    \begin{center}
	\begin{tabular}{p{107mm} p{9cm}}
	    \begin{minipage}{9cm}
	      \begin{center}
	      Politechnika Warszawska \\
	      Wydział Elektroniki i~Technik Informacyjnych \\
	      Instytut Informatyki
	      \end{center}
	    \end{minipage}
	    &
	    \begin{minipage}{8cm}
	    \begin{flushleft}
	     \footnotesize
	      Rok akademicki 2013/2014
	    \vspace*{2.75\baselineskip}
	    \end{flushleft}
	    \end{minipage} \\
	\end{tabular}
        \vspace*{2.5\baselineskip}
	\par\vspace{\smallskipamount}
	\begin{center}
		\includegraphics[angle=0,scale=1]{img/logo_pw.png}
	\end{center}
	\vspace*{2\baselineskip}{\LARGE Praca dyplomowa inżynierska\par}
	\vspace{3\baselineskip}{\LARGE\strut Krzysztof Opasiak\par}
	\vspace*{2\baselineskip}{\huge\bfseries Rozproszony monitoring systemów komputerowych\par}

	\vspace*{7\baselineskip}
	\hfill\mbox{}\par\vspace*{\baselineskip}\noindent
	\begin{tabular}[b]{@{}p{3cm}@{\ }l@{}}
	    {\large\hfill } & {\large }
	\end{tabular}
	\hfill
	\begin{tabular}[b]{@{}l@{}}
	Opiekun pracy: \\[\smallskipamount]
	{\large dr inż. Piotr Gawkowski}
	\end{tabular}\par
	\vspace*{4\baselineskip}
    \begin{tabular}{p{\textwidth}}
    \begin{flushleft}
	\begin{minipage}{7cm}
	Ocena \dotfill
	\par\vspace{1.6\baselineskip}
	\dotfill
	\par\noindent
	\centerline{\footnotesize Podpis Przewodniczącego} \par
	\centerline{\footnotesize Komisji Egzaminu Dyplomowego}\par
	\end{minipage}
    \end{flushleft}
    \end{tabular}
    \end{center}}

    % Życiorys
    \newpage\thispagestyle{empty}
    \begin{tabular}{p{4cm} p{12cm}}
    \begin{minipage}{5cm}
    
    \includegraphics[height=5cm,width=4cm]{img/foto.jpg}
    \end{minipage}
    &
    \begin{minipage}{12cm}
    \begin{flushleft}
    \par\noindent\vspace{1\baselineskip}
    \begin{tabular}[h]{l l}
    {\normalsize\it Kierunek:} & {\normalsize \hspace{0.8cm} Informatyka}
    \end{tabular}\par\noindent\vspace{1\baselineskip}
    \begin{tabular}[h]{l l}
    {\normalsize\it Specjalność:}& {\normalsize \hspace{0.4cm} Inżynieria Systemów Informatycznych}
    \end{tabular}
    \par\noindent\vspace{1\baselineskip}
    \begin{tabular}[h]{l l}
    {\normalsize\it Data urodzenia:} & {\normalsize \hspace{4.6cm} 1990.12.28}
    \end{tabular}
    \par\noindent\vspace{1\baselineskip}
    \begin{tabular}[h]{l l}
    {\normalsize\it Data rozpoczęcia studiów:} & {\normalsize \hspace{2.8cm} 2010.10.01}
    \end{tabular}
    \par\noindent\vspace{1\baselineskip}
    \end{flushleft}
    \end{minipage}
    \end{tabular}
    \vspace*{1\baselineskip}
    \begin{center}
	{\large\bfseries Życiorys}\par\bigskip
    \end{center}

    \indent 
    Urodziłem się 28 grudnia 1990 roku w Koninie. Uczęszczałem
    kolejno do Szkoły Podstawowej numer 8 im. Powstańców
    Wielkopolskich w Koninie, a następnie Gimnazjum Towarzystwa
    Salezjańskiego w Koninie.

    \par
    \indent 
    W latach 2006-2010 uczęszczałem do Technikum w Zespole
    Szkół im. Mikołaja Kopernika w Koninie. W trakcie nauki w tej
    szkole dwukrotnie przyznano mi stypendium Prezesa Rady Ministrów
    za bardzo dobre wyniki w nauce oraz wzorowe zachowanie.  W roku
    2010 ukończyłem z wyróżnieniem szkołę średnią, a następnie zdałem
    maturę oraz egzamin zawodowy, uzyskując tytuł: Technik
    Teleinformatyk.

    \par
    \indent
    W październiku 2010 roku rozpocząłem studia stacjonarne pierwszego stopnia na Wydziale Elektroniki
    i Technik Informacyjnych na kierunku Informatyka.

    \vspace{2\baselineskip}
    \hfill\parbox{15em}{{\small\dotfill}\\[-.3ex]
    \centerline{\footnotesize podpis studenta}}\par
    \vspace{2\baselineskip}
    \begin{center}
 	{\large\bfseries Egzamin dyplomowy} \par\bigskip\bigskip
    \end{center}
    \par\noindent\vspace{1.5\baselineskip}
    Złożył egzamin dyplomowy w dn. \dotfill 20\_\_r
    \par\noindent\vspace{1.5\baselineskip}
    Z wynikiem \dotfill
    \par\noindent\vspace{1.5\baselineskip}
    Ogólny wynik studiów \dotfill
    \par\noindent\vspace{1.5\baselineskip}
    Dodatkowe wnioski i uwagi Komisji \dotfill
    \par\noindent\vspace{1.5\baselineskip}
    \dotfill

    % Streszczenie
    \newpage\thispagestyle{empty}
    \vspace*{2\baselineskip}
    \begin{center}
	{\large\bfseries Streszczenie}\par\bigskip
    \end{center}

    {\itshape Przedmiotem pracy jest rozwinięcie systemu rozproszonego
      monitorowania Icinga pozwalające na monitorowanie urządzeń mobilnych.

      \indent Przedstawiono przegląd dostępnych na rynku systemów
      monitorujących oraz wskazano na istotne różnice w charakterze
      klienta mobilnego i~statycznego.

      Pozwoliło to na zdefiniowanie wymagań stawianych przed
      projektowanym rozszerzeniem możliwości systemu Icinga. Opisano
      również jego architekturę i omówiono dostępne do niego dodatki.

      \indent W efekcie prac zaproponowano autorskie rozszerzenie
      funkcjonalności w oparciu o dodatek NSCA oraz wyspecyfikowano
      własny protokół komunikacyjny. Zaimplementowany w ramach pracy
      dodatek pozwala na bezpieczne przekazywanie do systemu Icinga
      danych monitorowanych na urządzeniach mobilnych.

      \indent Praca zwiera też opis wykonanej konfiguracji testowej
      zaprojektowanego systemu. Praktyczność systemu potwierdzono
      kilkoma istotnymi spostrzeżeniami poczynionymi na podstawie
      zgromadzonych w systemie monitorującym danych.}\vspace*{1\baselineskip}

    \noindent{\bf Słowa kluczowe}: {\itshape monitorowanie,
      monitorowanie rozproszone, Icinga, protokół komunikacyjny, bezpieczeństwo komunikacji.}
    \par
    \vspace{4\baselineskip}
    \begin{center}
	{\large\bfseries Abstract}\par\bigskip
    \end{center}
    \noindent{\bf Title}: {\itshape Distributed monitoring of computer
      systems.}\par
    \vspace*{1\baselineskip} {\itshape The subject of this thesis is
      development of distributed monitoring system Icinga which allows
      for monitoring of mobile devices.

      \indent Review and comparison of available on market monitoring
      systems has been introduced and significant differences between
      mobile and stationary client has been pointed out.

      This allowed to define requirements for development of Icinga
      system. Architecture of Icinga and some available addons has
      been also described.

      \indent As a result of works, new enhancement of functionality
      based on NSCA addon with own communication protocol has been
      proposed. Implemented in this thesis addon allow for secure data
      transfer of monitoring data from mobile devices to Icinga
      server.

      \indent This thesis contains also description of created
      according to project, test infrastructure. Expedience of system
      has been confirmed with few significant remarks based on analyse
      of gathered data.
    } \vspace*{1\baselineskip}

    \noindent{\bf Key words}: {\itshape monitoring, distributed
      monitoring, Icinga, communiaction protocol, communication security.}

\end{titlepage}

% ex: set tabstop=4 shiftwidth=4 softtabstop=4 noexpandtab fileformat=unix filetype=tex spelllang=pl,en spell:
