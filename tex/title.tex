
\begin{titlepage}
    % Strona tytułowa
    \vbox to\textheight{\hyphenpenalty=10000
    \begin{center}
	\begin{tabular}{p{107mm} p{9cm}}
	    \begin{minipage}{9cm}
	      \begin{center}
	      Politechnika Warszawska \\
	      Wydział Elektroniki i~Technik Informacyjnych \\
	      Instytut Informatyki
	      \end{center}
	    \end{minipage}
	    &
	    \begin{minipage}{8cm}
	    \begin{flushleft}
	     \footnotesize
	      Rok akademicki 2013/2014
	    \vspace*{2.75\baselineskip}
	    \end{flushleft}
	    \end{minipage} \\
	\end{tabular}
        \vspace*{2.5\baselineskip}
	\par\vspace{\smallskipamount}
	\begin{center}
		\includegraphics[angle=0,scale=1]{img/logo_pw.png}
	\end{center}
	\vspace*{2\baselineskip}{\LARGE Praca dyplomowa inżynierska\par}
	\vspace{3\baselineskip}{\LARGE\strut Krzysztof Opasiak\par}
	\vspace*{2\baselineskip}{\huge\bfseries Rozproszony monitoring systemów komputerowych\par}

	\vspace*{7\baselineskip}
	\hfill\mbox{}\par\vspace*{\baselineskip}\noindent
	\begin{tabular}[b]{@{}p{3cm}@{\ }l@{}}
	    {\large\hfill } & {\large }
	\end{tabular}
	\hfill
	\begin{tabular}[b]{@{}l@{}}
	Opiekun pracy: \\[\smallskipamount]
	{\large dr inż. Piotr Gawkowski}
	\end{tabular}\par
	\vspace*{4\baselineskip}
    \begin{tabular}{p{\textwidth}}
    \begin{flushleft}
	\begin{minipage}{7cm}
	Ocena \dotfill
	\par\vspace{1.6\baselineskip}
	\dotfill
	\par\noindent
	\centerline{\footnotesize Podpis Przewodniczącego} \par
	\centerline{\footnotesize Komisji Egzaminu Dyplomowego}\par
	\end{minipage}
    \end{flushleft}
    \end{tabular}
    \end{center}}

    % Życiorys
    \newpage\thispagestyle{empty}
    \begin{tabular}{p{4cm} p{12cm}}
    \begin{minipage}{5cm}
    
    \includegraphics[height=5cm,width=4cm]{img/foto.jpg}
    \end{minipage}
    &
    \begin{minipage}{12cm}
    \begin{flushleft}
    \par\noindent\vspace{1\baselineskip}
    \begin{tabular}[h]{l l}
    {\normalsize\it Kierunek:} & {\normalsize \hspace{0.8cm} Informatyka}
    \end{tabular}\par\noindent\vspace{1\baselineskip}
    \begin{tabular}[h]{l l}
    {\normalsize\it Specjalność:}& {\normalsize \hspace{0.4cm} Inżynieria Systemów Informatycznych}
    \end{tabular}
    \par\noindent\vspace{1\baselineskip}
    \begin{tabular}[h]{l l}
    {\normalsize\it Data urodzenia:} & {\normalsize \hspace{4.6cm} 1990.12.28}
    \end{tabular}
    \par\noindent\vspace{1\baselineskip}
    \begin{tabular}[h]{l l}
    {\normalsize\it Data rozpoczęcia studiów:} & {\normalsize \hspace{2.8cm} 2010.10.01}
    \end{tabular}
    \par\noindent\vspace{1\baselineskip}
    \end{flushleft}
    \end{minipage}
    \end{tabular}
    \vspace*{1\baselineskip}
    \begin{center}
	{\large\bfseries Życiorys}\par\bigskip
    \end{center}

    \indent 
    Urodziłem się 28 grudnia 1990 roku w Koninie. Uczęszczałem
    kolejno do Szkoły Podstawowej numer 8 im. Powstańców
    Wielkopolskich w Koninie, a następnie Gimnazjum Towarzystwa
    Salezjańskiego w Koninie.

    \par
    \indent 
    W latach 2006-2010 uczęszczałem do Technikum w Zespole
    Szkół im. Mikołaja Kopernika w Koninie. W trakcie nauki w tej
    szkole dwukrotnie przyznano mi stypendium Prezesa Rady Ministrów
    za bardzo dobre wyniki w nauce oraz wzorowe zachowanie.  W roku
    2010 ukończyłem z wyróżnieniem szkołę średnią, a następnie zdałem
    maturę oraz egzamin zawodowy uzyskując tytuł Technik
    Teleinformatyk.

    \par
    \indent
    W październiku 2010 roku rozpocząłem studia stacjonarne pierwszego stopnia na Wydziale Elektroniki
    i Technik Informacyjnych na kierunku Informatyka.

    \vspace{2\baselineskip}
    \hfill\parbox{15em}{{\small\dotfill}\\[-.3ex]
    \centerline{\footnotesize podpis studenta}}\par
    \vspace{2\baselineskip}
    \begin{center}
 	{\large\bfseries Egzamin dyplomowy} \par\bigskip\bigskip
    \end{center}
    \par\noindent\vspace{1.5\baselineskip}
    Złożył egzamin dyplomowy w dn. \dotfill 20\_\_r
    \par\noindent\vspace{1.5\baselineskip}
    Z wynikiem \dotfill
    \par\noindent\vspace{1.5\baselineskip}
    Ogólny wynik studiów \dotfill
    \par\noindent\vspace{1.5\baselineskip}
    Dodatkowe wnioski i uwagi Komisji \dotfill
    \par\noindent\vspace{1.5\baselineskip}
    \dotfill

    % Streszczenie
    \newpage\thispagestyle{empty}
    \vspace*{2\baselineskip}
    \begin{center}
	{\large\bfseries Streszczenie}\par\bigskip
    \end{center}

    {\itshape Przedmiotem tej pracy jest system monitorowania
      rozproszone uwzględniający wymagania dotyczące monitorowania
      klienta mobilnego.

      \indent Na początku pracy wykonany został przegląd dostępnych na
      rynku systemów monitorujących. W~rozdziale~\ref{chap:Wymagania}
      przedstawiono charakterystykę klienta mobilnego
      i~statycznego. Zdefiniowano tam również wymagania stawiane przed
      projektowanym systemem. Rozdział~\ref{chap:Icinga} zawiera
      szczegółowy opis architektury systemu Icinga oraz omawia
      dostępne do niego dodatki.

      \indent Rozdział~\ref{chap:ProjektSystemu} zawiera opis
      proponowanej przez autora architektury rozwiązania. Zawiera on
      również definicję protokołu komunikacyjnego zaproponowanego
      w~tej pracy.

      \indent Rozdział \ref{chap:Implementacja} zawiera opis wykonanej
      implementacji dodatku umożliwiającego przekazywanie w~bezpieczny
      sposób danych z~urządzenia mobilnego do systemu Icinga.

      \indent W~końcowej części tej pracy znajduje się opis wykonanej
      w~ramach tej pracy konfiguracji testowej zaprojektowanego
      systemu. Zawarto tam również krótkie sprawozdanie oraz
      spostrzeżenia z~okresu testowego użytkowania
      systemu.}\vspace*{1\baselineskip}

    \noindent{\bf Słowa kluczowe}: {\itshape monitorowanie,
      monitorowanie rozproszone, Icinga, protokół komunikacyjny, bezpieczeństwo komunikacji.}
    \par
    \vspace{4\baselineskip}
    \begin{center}
	{\large\bfseries Abstract}\par\bigskip
    \end{center}
    \noindent{\bf Title}: {\itshape Distributed monitoring of computer
      systems.}\par
    \vspace*{1\baselineskip} {\itshape The subject of this thesis is
      distributed monitoring system which takes into account
      requirements of mobile devices.

      \indent In the beginning of this thesis review of few available
      on market monitoring systems has been presented. Chapter
      \ref{chap:Wymagania} describes characteristic of mobile and
      stationary clients. Requirements of proposed system has been
      also included there. Chapter \ref{chap:Icinga} contains detailed
      overview of Icinga and few of its addons.

      \indent Chapter \ref{chap:ProjektSystemu} contains description
      of proposed solution. Definition of communication protocol
      proposed in this thesis has also been included there.

      \indent Chapter \ref{chap:Implementacja} contains description of
      prepared implementation of prepared addon. This addon allow to
      transport data from mobile device to Icinga in a secure manner.

      \indent End of this thesis contains of description of test
      configuration of proposed system and short report with author remarks
      of use of the created system. } \vspace*{1\baselineskip}

    \noindent{\bf Key words}: {\itshape monitoring, distributed
      monitoring, Icinga, communiaction protocol, communication security.}

\end{titlepage}

% ex: set tabstop=4 shiftwidth=4 softtabstop=4 noexpandtab fileformat=unix filetype=tex spelllang=pl,en spell:
