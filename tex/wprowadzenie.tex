\chapter{Wprowadzenie}

Komputer stał się nieodzowną częścią współczesnej kultury. Praktycznie
każde gospodarstwo domowe posiada komputer wraz z~dostępem do internet
u. Urządzenia będące w~posiadaniu prywatnych właścicieli, bardzo
często są wykorzystywane do rozrywki lub innych czynności, których
niewykonanie nie pociąga za sobą żadnych konsekwencji. Znaczna jednak
część urządzeń znajduje się w posiadaniu dużych firm oraz ośrodków
badawczych. Komputery te zazwyczaj są połączone ze sobą w~sieć
prywatną - intranet. Do ich połączenia konieczna jest zarówno
rozbudowana struktura okablowania jak i~zestaw urządzeń
sieciowych. Wykorzystanie komputerów pozwala firmą na przyśpieszenie
prac, przez co ich zysk znacząco wzrasta. Brak możliwości używania
komputera, lub komunikacji poprzez infrastrukturę sieciową, niesie za
sobą poważne straty finansowe. Konieczne jest zatem zapewnienie
funkcjonowania całej infrastruktury, w~każdej chwili, gdy jest ona
potrzebna.

Urządzenia elektroniczne posiadają ograniczoną trwałość, przez co
istnieje niezerowe prawdopodobieństwo awarii każdego z~elementów sieci
firmowej. Ponadto należy pamiętać, iż na urządzeniach uruchamiane jest
oprogramowanie, które w~znaczącej większości nie jest pozbawione
błędów. Za awarię w danym systemie należy zatem uznać, nie tylko
fizyczne uszkodzenie urządzenia, lecz także sytuacje, w~której
użytkownik zostaje pozbawiony dostępu do danej aplikacji. Warto
również zauważyć, że użytkownik oczekuje od danej aplikacji
dostarczenia usług o~odpowiednim poziomie. Zatem należy również uznać,
za awarię, sytuację, gdy usługi świadczone użytkownikowi nie są na
satysfakcjonującym poziomie.

Użytkownik końcowy bardzo często nie jest w~stanie udzielić
precyzyjnej informacji o~usterce, którą może on obserwować. Bardzo
popularna jest sytuacja, w której użytkownik zgłasza, że nie działa
aplikacja z~której korzysta, natomiast faktyczną przyczyną błędu jest
awaria bazy danych lub komunikacji pomiędzy nimi. Indywidualna
diagnoza przy każdej awarii jest czasochłonna przez co czas do jej
usunięcia wydłuża się, powodując straty finansowe. Od wielu lat w celu
optymalizacji wykrywania i~obsługi awarii stosuje się systemy
monitorujące, które przedstawiają administratorowi w~wygodny sposób
stan wszystkich urządzeń oraz usług.

Zadaniem systemu monitorującego jest śledzenie stanu danego urządzenia
i~przedstawianie go administratorowi poprzez czytelny interfejs
użytkownika. Stan urządzenia może być rozumiany bardzo
szeroko. Istnieją systemy, które pozwalają na sprawdzanie, nie tylko
czy urządzenia jest włączone, lecz również jego szczegółowych
parametrów takich jak temperatura poszczególnych podzespołów czy
zużycie prądu. Można wyróżnić dwa podstawowe rodzaje monitorowania:

\begin{description}
\item[Monitorowanie aktywne] rodzaj monitorowania, w~którym system
  monitorujący cyklicznie wykonuje sprawdzenie danego urządzenia lub
  usługi
\item[Monitorowanie pasywne] rodzaj monitorowanie, w~którym status
  usługi lub urządzenia zgłaszany jest przez program zewnętrzny do
  systemu monitorującego
\end{description}

Każdy z~sposobów monitorowania ma zarówno swoje wady i~zalety. Wybór
metody monitorowania, zależy zatem od charakterystyki wartości, którą
monitorujemy. Jeśli wartość podlega, nieregularnym i~krótkotrwałym
zmianom stosuje się monitorowanie pasywne, aby nie było możliwe
pominięcie owego zdarzenia. Natomiast jeśli dana wartość posiada
charakterystykę zmieniającą się w~sposób ciągły, należy korzystać
wtedy z monitorowania aktywnego, które dokonuje próbkowania danej
wartości w~określonych odstępach czasu.

Sieci bardzo dużych przedsiębiorstw, nie posiadają płaskiej
struktury. Ze względów bezpieczeństwa bardzo często składa się ona
z~sieci wirtualnych, czy wręcz odizolowanych od siebie
podsieci. Ponadto sieci przedsiębiorstwa bardzo często zawierają
zapory ogniowe, które filtrują ruch pomiędzy sieciami. Ze względu na
fragmentacje sieci często nie jest możliwe zastosowanie prostego
systemu monitorowania opisanego wcześniej. Konieczne jest zatem użycie
rozproszonego systemu monitorowania.

Najprostszą realizacją rozproszonego systemu monitorowania jest użycie
monitorowania pasywnego do monitorowania wszystkich urządzeń i~usług,
które nie są widoczne z~sieci w~której uruchomiony jest system
monitorujący. Niestety wymaga to zmian w konfiguracji wszystkich
urządzeń i~uruchomienia na nim dodatkowego oprogramowania. Takie
zmiany mogą nie być możliwe, na prostych urządzeniach, których
kontrola odbywa się poprzez predefiniowany system producenta.

Możliwa jest również konfiguracja, wieloinstancyjnego systemu
monitorowania. W~każdej odizolowanej komórce sieci należy umieścić
instancję systemu, która będzie zbierała dane z~tej komórki sieci,
zarówno w sposób aktywny jak i~pasywny. Po wstępnym przetworzeniu
takich danych muszą one zostać zsynchronizowane pomiędzy instancjami,
a~następnie umieszczone w~instancji nadrzędnej lub innym miejscy
docelowym. Rozwiązanie to posiada liczne zalety i~jest bardzo często
stosowane. Dodatkowo niektóre z~systemów umożliwiają wymianę danych
pomiędzy instancjami bez konieczności istnienia wyróżnionej instancji
nadrzędnej.

Przedstawienie danych bieżących administratorowi jest często
niewystarczające. Precyzyjna diagnoza awarii w~możliwe krótkim czasie
od jej wystąpienia jest bardzo ważna. Jednak istotna jest również
możliwość analizy historycznych awarii, aby umożliwić wykrycie
potencjalnej awarii jeszcze przed jej wystąpieniem. Są dostępne na
rynku systemy, które pozwalają na gromadzenie danych o~odczytach
w~bazach danych. Kolejnym krokiem może być analiza takiej bazy danych
z~wykorzystaniem systemu eksperckiego, który wykaże odpowiednie
zależności i~na tej podstawie możliwe będzie wykrycie awarii jeszcze
przed jej wystąpieniem.

Współczesne korporacje posiadają nie tylko rozbudowaną infrastrukturę
sieciową, lecz również bardzo dużą liczbę urządzeń mobilnych takich
jak laptopy, tablety czy inne urządzenia specyficzne dla danej
firmy. Bardzo często okazuje się, że poprawne działanie tych urządzeń
wpływa znacząco na efektywność pracy, osób, które ich
używają. Monitorowanie takiego urządzenie jest zadaniem
nietrywialnym. Należy pamiętać, iż urządzenie mobilne może nie mieć
chwilowej możliwości komunikacji z~systemem monitorującym. Jeśli
przerwy w łączności występują stosunkowo często, to w~przypadku braku
późniejszej synchronizacji danych można doprowadzić, do fałszywych
predykcji systemu eksperckiego. Aby tego uniknąć konieczne jest
zapewnienie dostarczenia wyników wszystkich pomiarów do systemu, kiedy
tylko stanie się to możliwe. W przypadku klienta mobilnego niezwykle
istotna jest również kwestia bezpieczeństwa. Urządzenia takie często
nie pracują wewnątrz sieci formowej, lecz używają wielu różnych,
niezaufanych sieci do komunikacji. Dane zebrane podczas monitorowania
klienta mobilnego mogą zawierać tajemnice handlowe firmy. Konieczne
jest zatem zapewnienie zarówno poufności jak i~integralności danych
podczas synchronizacji.

Niestety, obecnie na rynku brak jest rozwiązań, które umożliwiałyby
monitorowanie klienta mobilnego. Ważne jest dostarczenie odpowiedniego
systemu, który pozwoli na kompleksowe monitorowanie wszystkich
urządzeń występujących w~firmie, zarówno mobilnych jak
i~statycznych. W~związku z~powyższym w~niniejszej pracy wykonano
rozbudowę popularnego systemu monitorowania, aby umożliwić
monitorowanie przy jego użyciu zarówno urządzeń statycznych jak
i~mobilnych.

Układ tej pracy jest następujący. Rozdział~\ref{chap:Systemy}
zawiera opis oraz porównanie dostępnych na rynków systemów
monitorowania oraz wykazuje ich braki. W~rozdziale XXX zdefiniowano
wymagania dla całego systemu monitorowania i~na ich podstawie wybrano
system, który podlegać będzie modyfikacjom. Rozdział XXX zawiera opis
systemu Icinga, który wybrano do rozbudowy, oraz wykazuje jakie
wymagania nie zostały przez ten system spełnione. W rozdziale XXX
przedstawiono opracowany projekt systemu monitorowania, zgodnego
z~przedstawionymi wymaganiami. Rozdział XXX natomiast zawiera opis
wykonanej implementacji oraz zaprojektowanego protokołu
komunikacyjnego. W~rozdziale XXX zawarto opis przebiegu testowania
wykonanego systemu, a~także przedstawiono sprawozdanie z~jego
użytkowania. Rozdział XXX natomiast, zawiera podsumowanie niniejszej
pracy, a~także wskazuje potencjalne możliwości rozwoju wykonanego
systemu.
