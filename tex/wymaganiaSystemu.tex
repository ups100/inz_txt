\chapter{Monitorowanie klienta mobilnego jako monitorowanie rozproszone}
\label{chap:Wymagania}

\section[Monitorowanie rozproszone][Monitorowanie rozproszone klientów
statycznych]{Monitorowanie rozproszone klientów statycznych}

Firmy działające obecnie na rynku posiadają bardzo rozbudowaną
infrastruktę informatyczna. Od bardzo wielu lat działy odpowiedzialne
za utrzymanie infrastruktury informatycznej prowadzą ciągły monitoring
zarówno urządzeń sieciowych jak i~serwerów oraz stacji roboczych
użytkowników. Bardzo wiele firm posiada również specjalistyczne
urządzenia, które również muszą być podłączone do sieci i~monitorowane
w celu zapewnienia ciągłości procesów biznesowych danej
firmy. Powyższe urządzenia rozumiane są jako klienty
statyczne. Urządzenia tego typu zazwyczaj pracują nieprzerwanie
i~posiadają dobrze zdefiniowaną hierarchię. Wzajemne relacje pomiędzy
tymi urządzeniami wynikają w~dużej mierze z~struktury sieci lecz mogą
również wynikać z~roli jaką pełnią one w~danej organizacji. Dzięki
monitorowaniu wszystkich urządzeń w~danej sieci systemy monitorujące
są w~stanie wspierać administratora wskazując z~bardzo dużym
prawdopodobieństwem miejsce wystąpienia awarii.

Sieć w~dużej firmie rzadko stanowi jedną całość. Zazwyczaj są to
segmenty sieci oddzielone zaporami lub w ogóle oddzielnie sieci LAN
lub VLAN. Taka separacja urządzeń pozwala na zwiększenie poziomu
bezpieczeństwa, lecz jednocześnie utrudnia monitorowanie całej
infrastruktury. Aby umożliwić monitorowanie całej sieci firmowej
wykorzystywane jest monitorowanie rozproszone. Można wyróżnić dwie
podstawowe konfiguracje monitorowania rozproszonego:

\begin{itemize}
\item Monitorowanie pasywne: Istnieje jedna, centralna instancja jądra
  monitorującego, do którego przesyłane są wyniki sprawdzeń
  poszczególnych usług. Każde urządzemoe samo monitoruje swoje usługi
  i~zgłasza rezultaty.
\item Wieloinstancyjny system monitorujący: Istnieje wiele instancji
  jądra monitorującego. Typowo, każda wydzielona część sieci posiada
  swoją instancję. Każda instancja może posiadać zarówno usługi
  monitorowane aktywnie jak i~pasywnie. Wyniki sprawdzeń przesyłane są
  następnie do jednej wybranej instancji, która gromadzi wszystkie
  dane.
\end{itemize}

%dorzucic tu jakies rysunki

Użycie monitorowania pasywnego dla wszystkich usług jest bardzo nie
wygodnie i~jednocześnie utrudnia konfiguracje, a~także pozbawia
administratora możliwosci używania niektórych mechanizmów dostępnych
wyłącznie dla urządzeń monitorowanych aktywnie. Ponadto wyniki
sprawdzeń pasywnych nie są akumulowane, lecz wysyłane odrazu po ich
uzyskaniu. Oznacza to, że jeśli pojawi się chwilowy brak połączenia
z~serwerem, to wpisy dziennika zostaną zgubione. W~przypadku, gdy
jedynym celem systemu jest monitorowanie dostępności danej usługi
zewnętrznej serwera, a nie jego parametrów wewnętrznych nie ma to
większego znaczenia.Kwestia ta staje się jednak istotna, gdy jednym z
zadań systemu, jest gromadzenia i analiza danych
historycznych. Wieleinstancyjny system minitorujący wymaga
zdecydowanie więcej zasobów jednak pozwala na osiągnięcie znacznie
wygodniejszego i~bardziej niezawodnego systemu. Ponadto dzięki takiej
konfiguracji nie ma potrzeby ingerencji w monitorowane serwery co
redukuje ich obciążenie, a~także zwiększa bezpieczeństwo. Warto
również wspomnieć, iż istnieją systemy, które dają możliwość
integracji wielu instancji jądra monitorującego. Dzięki temu
administrator danej sieci ma możliwość monitorowania i kofigurowania
wielu wielu instancji. Niestety rozwiązanie to posiada zamknięte
źródła i jego wykorzystanie wymaga zakupu licencji. Darmowe
rozwiązanie opera sie natomiast na istnieniu jednej centralnej
instancji jądra systemu monitorujacego, do kótrej przesyłane są w
ramach możliwości odczyty wykonane przez inne instancje.

\section[Monitorowanie rozproszone][Monitorowanie rozproszone klientów
mobilnych]{Monitorowanie rozproszone klientów mobilnych}

%\section[Charakterystyka problemu][Charakterystyka problemu]{Charakterystyka problemu}

Rosnąca w~ostatnich latach popularność technologii mobilnych
przyczyniła się do pojawienia się w firmach bardzo dużej liczby
urządzeń mobilnych, które wymagają zarówno zarządzania jak
i~monitorowania. Urządzenia mobilne są używane bardzo często przez
przedstawicieli handlowych, a~także przez menadżerów w~celu
umożliwienia wykonywania pracy poza obszarem firmy. Ponadto coraz
więcej firm świadczących zaawansowane technicznie usługi wyposaża
swoich pracowników w~bardzo drogi sprzęt, który wymaga ciągłego
monitorowania. Duże korporacje coraz częściej decydują się również na
wyposażenie swoich pracowników w smartfony lub tablety, które mają
ułatwić współpracę z firmą w trakcie podróży służbowych czy spotkań
z~klientami.

Klient mobilny posiada szereg cech, które znacząco odróżniają go od
klientów statycznych. Przedewszystkim należy zauważyć, że urządzenia,
o~których mowa bardzo często pracują poza obszarem firmy. Wynika
z~tego iż nie zawsze możliwe jest utrzymywanie takich urządzeń
w~wirtualnej sieci prywatnej, gdyż urządzenie może znaleźć się
w~obszarze, gdzie nie ma dostępu do internetu. Ponadto nie zawsze
konieczne jest, aby urządzenia mobilne pracowały podłączone do sieci
firmowej, gdyż dla użytkownika często wymagany jest jedynie dostęp do
internetu i inne funkcje tego urządzenia. Warto więc zdać sobie
sprawę, że urządzenia te są często narażone na dostęp do sieci,
o~bardzo niskim poziomie zaufania i wielu zagrożeniach. Oznaczia to
w~szczególności, iż urządzenie mobilne zazwyczaj posiada zmienny adres
IP, który rzadko jest adresem globalnym. Również struktura sieci,
z~której korzystają klienty mobilne jest dynamiczna i~znajduje się
poza obszarem monitorowania administratorów danego
przedsiębiorstwa. Znacząca większość klientów mobilnych dzięki
kontaktom z~sieciom poza firmową posiada, w~przeciwieństwie do
klientów statycznych, możliwość synchronizacji swojego czasu czy to
z~serwerami czasu światowego, czy też z~sieci GSM.

Należy również zwrócić uwagę na duże rozproszenie klientów
mobilnych. W~przeciwieństwie do klientów statycznych, którzy zazwyczaj
pracują w~pewnych grupach lub fragmentach sieci, klienty mobilne są
zazwyczaj rozpatrywane pojedyńczo. Większość klientów mobilnych
operuje w~pełni samodzielnie, zatem grupa liczność grupy klientów
wynosi zazwyczaj~1. Powoduje to, że w przeciwieństwie do klientów
statycznych gdzie grup koniecznych do wydzielenia było zazwyczaj kilka
lub kilkanaście, w~przypadku klientów mobilnych takich grup może być
kilkaset lub nawet kilka tysięcy. Warto również dostrzec różnice
w~zasilaniu. Klienty mobilne zazwyczaj posiadają własne zasilanie,
przez co każda operacja wykonywana na nim nie tylko spowalnia jego
działanie, lecz również zmniejsza jego czas pracy pomiędzy
ładowaniami. Przenośność klienta mobilnego zmienia również jego
stopień bezpieczeństwa. Urządzenia mobilne stosunkowo często są
gubione lub kradzione, co nie było możliwe w~przypadku klientów
statycznych. W~związku z~możliwością utraty urządzenia, nie powinno
sie na nim przechowywać tajnych danych, dzięki którym możnaby
skompromitować cały system z~którego korzysta klient.

Klient mobilny znacznie różni się swoją charakterystyką od klienta
statycznego. Różni się również rodzaj monitorowanych
usług. W~przypadku klientów statycznych znaczna część wysiłków jest
skierowana na pomiar usług świadczonych przez dany system dla innych
systemów. Natomiast w~przypadku klientów mobilnych znacznie większy
nacisk jest położony na monitorowanie parametrów wewnętrznych danego
klienta.

\section[Wymagania][Wymagania systemu monitorowania klientów
mobilnych]{Wymagania systemu monitorowania klientów mobilnych}

Klient mobilny posiada zdecydowanie odmienną charakterystykę niż
klient statyczny. Dokonano zatem głębokiej analizy, jakie wymagania
należały spełnić, aby dostarczyć system, który sprosta oczekiwaniom
administratów.

Odbiorcą systemu mają być duże firmy i korporacje, które posiadają
bardzo rozbudowaną sieć wewnątrz firmy, a ponadto udostępniają swoim
pracownikom urządzenia mobilne różnej klasy. Wsród tych urządzeń
znajdują się przedewszystkim telefony oraz tablety z systemem
operacyjnym Android oraz Windows Phone, a także liczne laptopy
wyposażone w system Windows lub Linux. Konieczne jest zatem, aby
system pozwalał na monitorowanie każdej z wspomnianych platform. Duże
firmy oraz korporacje, zazwyczaj posiadają już oprogramowanie służące
do minitorowania swojej infrastruktury sieciowej. Aby umożliwić
administratorom łatwe zarządzanie oraz monitorowanie zarówno klientami
mobilnymi jak i statycznymi, należy zapewnić integrację systemów
monitorowania obu kategorii klientów. Dane odczytywane na urządzeniu
mobilnym mogą zawierać zarówno dane prywatne pracownika, jak i
tajemnice handlowe firmy. Oba te rodzaje danych nalezą do kategorii
poufnych i powiny być należycie chronione. Ponieważ urządzenie mobilne
będzie pracowało często poza siecią firmową, podczas tworzenia systemu
należy zwrócić szczególną uwagę na kwestię bezpieczeństwa przesyłanych
danych. Ponieważ system, musi przesyłać dane poprzez sieć publiczną,
konieczne jest również zapewnienie odporności systemu na ataki
zewnętrzne oraz na próby przekazywania sfałszowanych danych do
systemu. Wszystkie wymagania stawiane przed omawianym systemem zostały zebrane w \ref{tab:Wymagania}.

\begin{longtable}[c]{|c||p{3.5cm}|p{9cm}|}
\caption[Wymagania systemu monitorowania klienta mobilnego]{Wymagania systemu monitorowania klienta mobilnego} \label{tab:Wymagania} \\ 
  \hline
  Kod & \multicolumn{1}{c|}{Nazwa} & \multicolumn{1}{c|}{Opis} \tabularnewline
  \hline \hline
  \endfirsthead

  \multicolumn{3}{c}%
  {{\tablename\ \thetable{} -- Kontynuacja z~poprzedniej strony}} \\
  \hline
  Kod & \multicolumn{1}{c|}{Nazwa} & \multicolumn{1}{c|}{Opis} \tabularnewline
  \hline \hline
  \endhead

  \hline \multicolumn{3}{|r|}{{Kontynuacja na następnej stronie}} \\ \hline
  \endfoot

  \hline\hline
  \endlastfoot
  
  W1 & \raggedright{Spójność danych} & \raggedright{System musi zapewnić, że wpisy dziennika nie zostaną zgubione. System musi zapewniać spójność danych pomiędzy serwerem, a~klientem mobilnym.} \tabularnewline
  \hline

  W2 & Integralności & \raggedright{System musi zapewnić, że wpisy dziennika dostarczone do serwera nie zostały w~żaden sposób zmodyfikowane lub dodane.} \tabularnewline
  \hline

  W3 & Autentyczność & \raggedright{System musi zapewnić, że odebrane dane pochodzą od uprawionego klienta.} \tabularnewline
  \hline
  
  W4 & Poufność & \raggedright{System musi zapewniać poufność danych przesyłanych od klienta poprzez szyfrowanie.} \tabularnewline
  \hline

  W5 & \raggedright{Dodawanie algorytmów} & \raggedright{System musi być niezależny od algorytmu kryptograficznego stosowanego podczas przesyłania danych. Ponadto system musi umożliwać dodawanie w prosty sposób nowych algorytmów kryptograficznych.} \tabularnewline
  \hline

  W6 & \raggedright{Uwierzytelnienie klienta} & \raggedright{System musi zapewnić możliwość uwierzytelnienia klienta.} \tabularnewline
  \hline

  W7 & \raggedright{Wymienne algorytmy uwierzytelnienia klienta} & \raggedright{System musi być niezależny od algorytmu uwierzytelnienia kliena. Ponadto system musi umożliwaić dodanie w prosty sposób nowych algorytmów uwierzytelnienia klienta.} \tabularnewline
  \hline

  W8 & \raggedright{Uwierzytelnienie serwera} & \raggedright{System musi zapewniać, iż wpisy dziennika zostaną przesłane tylko do wyznaczonego, uprawnionego serwera.} \tabularnewline
  \hline

  W9 & \raggedright{Odporność na zgubienie urządzenia} & \raggedright{Systes musi być odporny na zgubienie urządzenie. Oznacza to iż zgubienie urządzenia nie może powodować kompromitacji całego systemu.} \tabularnewline
  \hline

  W10 & \raggedright{Dostarczanie w wiele miejsc} & \raggedright{System musi umożliwiać przekazywanie danych do wielu podsystemów monitorujących, bez konieczności ich retransmisjii z~klienta mobilnego.} \tabularnewline
  \hline

  W11 & \raggedright{Reguły definiowane dla każdego klienta} & \raggedright{System musi umożliwiać definowanie reguł dotyczących miejsc przeznaczenia dla każdego klienta indywidualnie.} \tabularnewline
  \hline

  W12 & \raggedright{Oszczędność pasma} & \raggedright{System powinien minimalizować ilość przesyłanych danych. Ponadto powinien skrócić do minimum czas oczekiwania na potwierdzenie przetworzenia przesłanych danych.} \tabularnewline
  \hline

  W13 & \raggedright{Integracja z~istniejącymi systemami} & \raggedright{System monitoringu klienta mobilnego musi mieć możliwość integracji i~współpracy z~istniejącymi systemami monitorowania klienta statycznego.} \tabularnewline
  \hline

  W14 & \raggedright{Analiza danych bierzących} & \raggedright{System musi umożliwiać prezentację oraz analizę danych bierzących, a~także posiadać możliwość reagowania na wystąpienie zdefiniowanych przez użytkownika zdarzeń.} \tabularnewline
  \hline

  W15 & \raggedright{Analiza danych historycznych} & \raggedright{System musi umożliwiać analizę zadanych danych historycznych włączając w~to ich graficzną reprezentację.} \tabularnewline
  \hline

  W16 & \raggedright{Kontrola danych wejściowych} & \raggedright{System musi prowadzić kontrolę danych wejściowych od klientów. Konieczne jest aby system umożliwiał definiowanie jakie dane mogą być dostarczane przez jakich klientów.} \tabularnewline
  \hline

  W17 & \raggedright{Łatwość dodawania nowych sprawdzeń} & \raggedright{System musi umożliwiać dodawanie w łatwy sposób możliwości monitorowania nowych usług i~parametrów.} \tabularnewline
  \hline

  W18 & \raggedright{Klient dla platformy Android} & \raggedright{System musi udostępniać klienta pozwalającego na monitorowanie urządzeń opartych na platformie Android} \tabularnewline
  \hline

  W19 & \raggedright{Klient dla platformy Windows Phone} & \raggedright{System musi udostępniać klienta pozwalającego na monitorowanie urządzeń opartych na platformie Windows Phone} \tabularnewline 
  \hline

  W20 & \raggedright{Klient dla platformy Windows 8} & \raggedright{System musi udostępniać klienta pozwalającego na monitorowanie urządzeń opartych na platformie Windows 8} \tabularnewline
  \hline

  W21 & \raggedright{Klient dla platformy Linux} & \raggedright{System musi udostępniać klienta pozwalającego na monitorowanie urządzeń opartych na platformie Linux} \tabularnewline
  \hline
\end{longtable}
